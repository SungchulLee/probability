\documentclass[12pt]{article}%report, article
%\documentstyle[12pt,leqno]{article}

\textwidth=15.5cm \textheight=21.6cm \topmargin=-0.5cm
\oddsidemargin=0.05cm

\newcommand{\bbA}{{\bf A}}
\newcommand{\bba}{{\bf a}}
\newcommand{\bbB}{{\bf B}}
\newcommand{\bbb}{{\bf b}}
\newcommand{\bbC}{{\bf C}}
\newcommand{\bbc}{{\bf c}}
\newcommand{\bbD}{{\bf D}}
\newcommand{\bbd}{{\bf d}}
\newcommand{\bbE}{{\bf E}}
\newcommand{\bbe}{{\bf e}}
\newcommand{\bbI}{{\bf I}}
\newcommand{\bbi}{{\bf i}}
\newcommand{\bbJ}{{\bf J}}
\newcommand{\bbj}{{\bf j}}
\newcommand{\bbK}{{\bf K}}
\newcommand{\bbk}{{\bf k}}
\newcommand{\bbP}{{\bf P}}
\newcommand{\bbp}{{\bf p}}
\newcommand{\bbQ}{{\bf Q}}
\newcommand{\bbq}{{\bf q}}
\newcommand{\bbT}{{\bf T}}
\newcommand{\bbt}{{\bf t}}
\newcommand{\bbU}{{\bf U}}
\newcommand{\bbu}{{\bf u}}
\newcommand{\bbV}{{\bf V}}
\newcommand{\bbv}{{\bf v}}
\newcommand{\bbW}{{\bf W}}
\newcommand{\bbw}{{\bf w}}
\newcommand{\bbX}{{\bf X}}
\newcommand{\bbx}{{\bf x}}
\newcommand{\X}{{\cal X}}
\newcommand{\bbY}{{\bf Y}}
\newcommand{\bby}{{\bf y}}
\newcommand{\bbZ}{{\bf Z}}
\newcommand{\bbz}{{\bf z}}
\newcommand{\0}{{\bf 0}}
\newcommand{\R}{{\bf R}}
\newcommand{\txi}{\bar{\xi}}
\def\Comment#1{ \marginpar{$\bullet$\quad{\tiny #1}}}


\usepackage{graphics,graphicx,amsmath,float,color,subfigure,enumerate,booktabs}
%\usepackage[tiling]{pst-fill}
\usepackage[dvips]{xy}
\usepackage{tikz}
\usetikzlibrary{matrix}
\input{rgb}
\xyoption{all}


\newcommand{\ra}[1]{\renewcommand{\arraystretch}{#1}}


\begin{document}
%controls the margin
\baselineskip=6.0mm








%Ignore some parts of statements
\newcommand{\ignore}[1]{}{}





%Equation numbers contain section number
\renewcommand{\theequation}{\arabic{section}.\arabic{equation}}





%Activate this if I want to display eq number only
\newcommand{\lbl}{\label}

%Activate this if I want to display eq number and text number, too
%\newcommand{\lbl}[1]{\hspace{1cm} \underline{({#1})} \label{#1}}





%Call this eq numbers in text
\newcommand{\eq}[1]{$(\ref{#1})$}


\newcommand{\f}{\frac}    


%Short for Greek letters
\newcommand{\al}{\alpha}                         %\al=\al
\newcommand{\bt}{\beta}                          %\bt=w
\newcommand{\ga}{\gamma}                         %\ga=\gamma
\newcommand{\Ga}{\Gamma}                         %\Ga=\Gamma
\newcommand{\de}{\delta}                         %\de=\delta
\newcommand{\De}{\Delta}                         %\De=\Delta
\newcommand{\ep}{\epsilon}                       %\ep=\epilon
\newcommand{\ve}{\varepsilon}                    %\ve=\varepsilon
\newcommand{\la}{\lambda}                        %\la=\lambda
\newcommand{\La}{\Lambda}                        %\La=\Lambda
\newcommand{\ro}{\rho}                           %\ro=\rho
\newcommand{\ta}{\tau}                           %\ta=tau
%\newcommand{\th}{\theta}                         %\th=\theta
\newcommand{\si}{\sigma}                         %\si=\sigma
\newcommand{\Si}{\Sigma}                         %\si=\sigma
\newcommand{\om}{\omega}                           %\ro=\rho
\newcommand{\Om}{\Omega}                           %\ta=tau





%Short for equation array and for equations
%%%%%%%%%%%%%%%%%%%%%%%%%%%%%%%%%%%%
\newcommand{\be}{\begin{equation}}               %\be=\begin{equation}
\newcommand{\ee}{\end{equation}}                 %\ee=\end{equation}
%%%%%%%%%%%%%%%%%%%%%%%%%%%%%%%%%%%%
\newcommand{\bea}{\begin{eqnarray}}              %\bea=\begin{eqnarray}
\newcommand{\eea}{\end{eqnarray}}                %\eea=\end{eqnarray}
%%%%%%%%%%%%%%%%%%%%%%%%%%%%%%%%%%%%
\newcommand{\bean}{\begin{eqnarray*}}            %\beq=\begin{eqnarray*}
\newcommand{\eean}{\end{eqnarray*}}              %\eeq=\end{eqnarray*}
%%%%%%%%%%%%%%%%%%%%%%%%%%%%%%%%%%%%
%\newcommand{\beq}{\begin{eqnarray*}}            %\beq=\begin{eqnarray*}
%\newcommand{\eeq}{\end{eqnarray*}}              %\eeq=\end{eqnarray*}
%%%%%%%%%%%%%%%%%%%%%%%%%%%%%%%%%%%%
\newcommand{\ba}{\begin{array}}                  %\ba=\begin{array}
\newcommand{\ea}{\end{array}}                    %\ea=\end{array}
%%%%%%%%%%%%%%%%%%%%%%%%%%%%%%%%%%%%
\newcommand{\nn}{\nonumber}                      %\nn=\nonumber
\newcommand{\mb}{\mbox}                          %\mb=\mbox





%\newcommand{\ra}{\rightarrow}                    %\ra=\rightarrow
\newcommand{\Ra}{\Rightarrow}                    %\ra=\rightarrow
\newcommand{\ua}{\uparrow}   
\newcommand{\da}{\downarrow}   
\newcommand{\Lra}{\Leftrightarrow}          %\llra=\longleftrightarrow
\newcommand{\llra}{\longleftrightarrow}          %\llra=\longleftrightarrow

\newcommand{\stac}{\stackrel}                    %\stac=\stackrel
\newcommand{\noin}{\noindent}                    %\noin=\noindent

%Symbol for the end of the proof
\newcommand{\qed}{\nobreak\quad\vrule width6pt depth3pt height10pt}

\newcommand{\ngi}{n \ra \infty}

%heading
%\pagestyle{myheadings} \markright{Rooted edges of a minimal directed spanning tree on random points}
\pagestyle{myheadings} \markright{[Homework] Distributions related to the Poisson point process}

\thispagestyle{plain}


\begin{center}
{\Large\bf [Homework] Distributions related to the Poisson point process} 
\end{center}






\begin{enumerate}



\item
The county hospital is located at the center of a square whose sides are 4 km wide. If an accident occurs within this square, then the hospital sends out an ambulance. The road network is rectangular, so the travel distance from the hospital, whose coordinates are $(0, 0)$, to the point $(x, y)$ is $|x| + |y|$. If an accident occurs at a point that is uniformly distributed in the square, find the expected travel distance of the ambulance.
\\
{\color{blue}{\bf Sol.}}
We can assume that the coordinates $X$ and $Y$ of the accident are independent and uniformly distributed over $(-2, 2)$. Hence the expected travel distance of the ambulance is
 $$E(|X| + |Y |) = E(|X|) + E(|Y |)=2\cdot \int_{-2}^{2}|x|\cdot\f{1}{3} dx = \f{8}{3}$$
 
 

\item
Buses arrive at a specified stop at 15-minute intervals starting at 7 A.M. 
That is, they arrive at 7, 7:15, 7:30, 7:45, and so on. 
If a passenger arrives at the stop at a time that is uniformly distributed between 7 and 7:30, find the probability that he waits
\begin{enumerate}
\item
less than 5 minutes for a bus.
\\
{\color{blue}{\bf Sol.}}
$X$ = number of minutes past $7$ the passenger arrives.\\
$X$ is a uniform random variable over the interval $(0,30)$.
$$P(10 < X <15) + P(25 < X <30) = \int_{10}^{15} \f{1}{30} dx +\int_{25}^{30} \f{1}{30} dx =\f{1}{3}$$

 




\item
more than10 minutes for a bus.
\\
{\color{blue}{\bf Sol.}}
$$P(0 < X <5) + P(15 < X <20) = \int_{0}^{5} \f{1}{30} dx +\int_{15}^{20} \f{1}{30} dx =\f{1}{3}$$

\end{enumerate}








\item
The joint PDF $f_{X,Y}(x,y)$ of $X$ and $Y$ is given by
$$
f(x,y)=12xy(1-x),\ \ \ \ \ \mb{for $0<x<1$ and $0<y<1.$}
$$
\begin{enumerate}
\item
Find the PDF $f_{X}(x)$ of $X$ and the PDF $f_{Y}(y)$ of $Y$.
\\
{\color{blue}{\bf Sol.}}
$$f_X(x)=\int_0^1 f(x,y)dy=6x-6x^2$$
$$f_Y(y)=\int_0^1 f(x,y)dx=2y$$




\item
Are $X$ and $Y$ independent?
\\
{\color{blue}{\bf Sol.}}
Since $f(x,y) = f_X(x)f_Y(y)$, $X$ and $Y$ are independent.



\item
Calculate $EX$, $EY$, $Var(X)$, $Var(Y)$, $Cov(X,Y)$.
\\
{\color{blue}{\bf Sol.}}
(c)

$$EX=\int_0^1 xf_X(x)dx=\f{1}{2},~~ EY=\int_0^1 yf_Y(y)dy=\f{2}{3}$$
$$EX^2=\int_0^1 x^2f_X(x)dx=\f{3}{10}, ~~ EY^2=\int_0^1 y^2f_Y(y)dy=\f{1}{2}$$
$$VarX=EX^2-(EX)^2=\f{1}{20}, ~~ VarY=EY^2-(EY)^2=\f{1}{18}$$
$$EXY=\int_0^1 \int_0^1   xyf(x,y)dxdy=\f{1}{3}$$
$$Cov(X,Y)=EXY-EX\cdot EY=0$$



\end{enumerate}






\item
The joint PDF $f(x,y)$ of $X$ and $Y$ is given by
$$
f(x,y)=x+y,\ \ \ \ \ \mb{for $0\le x\le 1$ and $0\le y\le 1.$}
$$
\begin{enumerate}
\item
Are $X$ and $Y$ independent?
\\
{\color{blue}{\bf Sol.}}
 Since $f(x,y) \neq f_X(x)f_Y(y)$, $X$ and $Y$ are not independent.


\item
Find the PDF $f_{X}(x)$ of $X$.
\\
{\color{blue}{\bf Sol.}}

$$ f_X(x)= \int_0^1 f(x,y)dy=x+\f{1}{2} $$
$$ f_Y(y)= \int_0^1 f(x,y)dx=y+\f{1}{2} $$

\item
Calculate $P(X+Y<1)$.
\\
{\color{blue}{\bf Sol.}}
$$ P(X+Y<1)= \int_0^1 \int_0^{1-y} f(x,y)dxdy=\f{1}{3} $$

\end{enumerate}





\item
The joint PDF $f(x,y)$ of $X$ and $Y$ is given by
$$
f(x,y)=xe^{-(x+y)},\ \ \ \ \ \mb{for $x>0$ and  $y>0$.}
$$
\begin{enumerate}
\item
Identify the distribution of $X$ and report its mean and variance.
\\
{\color{blue}{\bf Sol.}}
$$f_X(x)=\int_0^\infty f(x,y)dy=xe^{-x}$$
$$X \sim \Ga(2,1)$$
$$EX=\f{n}{\la}=2, VarX=\f{n}{\la^2}=2$$

 




\item
Are $X$ and $Y$ independent?
\\
{\color{blue}{\bf Sol.}}

$$f_Y(y)=\int_0^\infty  f(x,y)dx=e^{-y}$$
$$Y \sim \exp(1),EX=\f{1}{\la}=1, VarY=\f{1}{\la^2}=1$$
Since $f(x,y) = f_X(x)f_Y(y)$, $X$ and $Y$ are independent.

\item
Identify the distribution of $X+Y$ and report its mean and variance.
\\
{\color{blue}{\bf Sol.}}
Since $X$ and $Y$ are independent, $X+Y \sim \Ga(3,1)$
$$E(X+Y)=EX+EY= 3$$
$$Var(X+Y)=VarX+VarY+2Cov(X,Y)=3$$

 



\end{enumerate}


\item
Show that
$$
B(n,p)
*
B(m,p)
=
B(n+m,p)
$$
\\
{\color{blue}{\bf Sol.}}
\bea
f_Z(z)
&=&f_{X+Y}(z)\nn\\
&=&\sum_{x=0}^z f_X(x)f_Y(z-x)\nn\\
&=&\sum_{x=0}^z {n_1 \choose x}p^x(1-p)^{n_1-x} {n_2 \choose z-x}p^{z-x}(1-p)^{n_2-(z-x)}\nn\\
&=&p^z(1-p)^{n_1+n_2-z}\sum_{x=0}^z {n_1 \choose x}{n_2 \choose z-x}\nn\\
&=&{n_1+n_2 \choose z}p^z(1-p)^{n_1+n_2-z} \nn
\eea

\item
Show that
$$
Exp(\la)
*
Exp(\la)
=
\Ga(2,\la)
$$
\\
{\color{blue}{\bf Sol.}}
Let $X, Y$, and $Z = X + Y$ denote the relevant random variables, and $f_X, f_Y$,
and $f_Z$ their densities. Then
$$
f_X(x)=f_Y(x)=\left\{\ba{ll}
\la e^{-\la x} & x\ge0  \\
0 &\mb{otherwise}\
\ea\right.
$$

for $z>0$
\bea
f_Z(z)
&=&\int_{-\infty}^{\infty} f_X(z-y)f_Y(y)\nn\\
&=&\la^2ze^{-\la z}\nn\\
&=&\f{1}{\Ga(2)}\la\cdot(\la z)^{2-1}e^{-\la z}\nn
\eea

Therfore, $$Exp(\la)*Exp(\la)=\Ga(2,\la)$$


\item
The lifetime in hours of an electronic tube is a random variable having a PDF given by
$$
f(x)=xe^{-x},
\quad
x>0
$$
Compute the expected lifetime of such a tube.
\\
{\color{blue}{\bf Sol.}}


$$f(x)=xe^{-x}=\f{1}{\Ga(2)}\cdot1\cdot(1\cdot x)^{2-1}e^{-1\cdot x}$$
$$X \sim \Ga(\al =2, \la=1) $$
$$EX=\f{n}{\la}=2$$









\item 
When I enter the bank, there are already two people in line waiting for the service and I join the queue. 
In the bank there are four service desks and we assume the service time is iid $Exp(\la_1)$, $\la_1 = 2$ (in minutes). 
After I got serviced at bank, I visit the post office.
When I enter the post office,  there are already three people in line waiting for the service and I join the queue. 
In the post office
there are two service desks and we assume the service time is iid $Exp(\la_2)$, $\la_2 = 4$ (in minutes). 
Let $F$ be the fraction of waiting time in post office
among the total waiting time in both the bank and the post office.
Calculate
the mean and variance of $F$.
\\
{\color{blue}{\bf Sol.}}
$X_i$ : the $i(i=1,2)$th person waiting time in the bank, then $X_i \sim Exp(4\la_1)$\\
$X_3$ : my waiting time in the bank, then $X_3 \sim Exp(4\la_1)$\\
$X=X_1+X_2+X_3 \sim \Ga(3,4 \la_1)$\\


$Y_i$ : the $i(i=1,2,3)$th person waiting time in the post office, then $Y_i \sim Exp(2\la_2)$\\
$Y_4$ : my waiting time in the post office, then $Y_3 \sim Exp(2\la_2)$\\
$Y=Y_1+Y_2+Y_3+Y_4 \sim \Ga(4,2 \la_2)$



$F=\f{Y}{X+Y}$ is the fraction of waiting time spent in post office
among the total waiting time in both the bank and the post office. Then
$$F \sim Beta(\beta,\al)$$



$$EF=\f{\beta}{\al+\beta}=\f{4}{7}$$
$$VarF=\f{\al\beta}{(\al+\beta)^2(\al+\beta+1)}=\f{3}{98}$$



\item
Let $T$ be the inter arrival time containing 05/05/2013 of $PPP(2)$.
Find its mean and variance.
\\
{\color{blue}{\bf Sol.}}
$T \sim \Ga(2,2)$
$$ET=\f{n}{\la}=1,~~ VarT=\f{n}{\la^2}=\f{1}{2}$$








\end{enumerate}





\begin{center}
{\Large\bf [Extra]} 
\end{center}






\begin{enumerate}
\item
An accident occurs at a point $X$ that is uniformly distributed on a road of length $L$. 
At the time of the accident, an ambulance is at a location $Y$ that is also uniformly
distributed on the road. Assuming that $X$ and $Y$ are independent, find the expected
distance between the ambulance and the point of the accident.





\end{enumerate}



\end{document}
