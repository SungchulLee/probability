\documentclass[12pt]{article}%report, article
%\documentstyle[12pt,leqno]{article}

\textwidth=15.5cm \textheight=21.6cm \topmargin=-0.5cm
\oddsidemargin=0.05cm

\newcommand{\bbA}{{\bf A}}
\newcommand{\bba}{{\bf a}}
\newcommand{\bbB}{{\bf B}}
\newcommand{\bbb}{{\bf b}}
\newcommand{\bbC}{{\bf C}}
\newcommand{\bbc}{{\bf c}}
\newcommand{\bbD}{{\bf D}}
\newcommand{\bbd}{{\bf d}}
\newcommand{\bbE}{{\bf E}}
\newcommand{\bbe}{{\bf e}}
\newcommand{\bbI}{{\bf I}}
\newcommand{\bbi}{{\bf i}}
\newcommand{\bbJ}{{\bf J}}
\newcommand{\bbj}{{\bf j}}
\newcommand{\bbK}{{\bf K}}
\newcommand{\bbk}{{\bf k}}
\newcommand{\bbP}{{\bf P}}
\newcommand{\bbp}{{\bf p}}
\newcommand{\bbQ}{{\bf Q}}
\newcommand{\bbq}{{\bf q}}
\newcommand{\bbT}{{\bf T}}
\newcommand{\bbt}{{\bf t}}
\newcommand{\bbU}{{\bf U}}
\newcommand{\bbu}{{\bf u}}
\newcommand{\bbV}{{\bf V}}
\newcommand{\bbv}{{\bf v}}
\newcommand{\bbW}{{\bf W}}
\newcommand{\bbw}{{\bf w}}
\newcommand{\bbX}{{\bf X}}
\newcommand{\bbx}{{\bf x}}
\newcommand{\X}{{\cal X}}
\newcommand{\bbY}{{\bf Y}}
\newcommand{\bby}{{\bf y}}
\newcommand{\bbZ}{{\bf Z}}
\newcommand{\bbz}{{\bf z}}
\newcommand{\0}{{\bf 0}}
\newcommand{\R}{{\bf R}}
\newcommand{\txi}{\bar{\xi}}
\def\Comment#1{ \marginpar{$\bullet$\quad{\tiny #1}}}


\usepackage{graphics,graphicx,amsmath,float,color,subfigure,enumerate,booktabs}
%\usepackage[tiling]{pst-fill}
\usepackage[dvips]{xy}
\usepackage{tikz}
\usetikzlibrary{matrix}
\input{rgb}
\xyoption{all}


\newcommand{\ra}[1]{\renewcommand{\arraystretch}{#1}}


\begin{document}
%controls the margin
\baselineskip=6.0mm








%Ignore some parts of statements
\newcommand{\ignore}[1]{}{}





%Equation numbers contain section number
\renewcommand{\theequation}{\arabic{section}.\arabic{equation}}





%Activate this if I want to display eq number only
\newcommand{\lbl}{\label}

%Activate this if I want to display eq number and text number, too
%\newcommand{\lbl}[1]{\hspace{1cm} \underline{({#1})} \label{#1}}





%Call this eq numbers in text
\newcommand{\eq}[1]{$(\ref{#1})$}


\newcommand{\f}{\frac}    


%Short for Greek letters
\newcommand{\al}{\alpha}                         %\al=\al
\newcommand{\bt}{\beta}                          %\bt=w
\newcommand{\ga}{\gamma}                         %\ga=\gamma
\newcommand{\Ga}{\Gamma}                         %\Ga=\Gamma
\newcommand{\de}{\delta}                         %\de=\delta
\newcommand{\De}{\Delta}                         %\De=\Delta
\newcommand{\ep}{\epsilon}                       %\ep=\epilon
\newcommand{\ve}{\varepsilon}                    %\ve=\varepsilon
\newcommand{\la}{\lambda}                        %\la=\lambda
\newcommand{\La}{\Lambda}                        %\La=\Lambda
\newcommand{\ro}{\rho}                           %\ro=\rho
\newcommand{\ta}{\tau}                           %\ta=tau
%\newcommand{\th}{\theta}                         %\th=\theta
\newcommand{\si}{\sigma}                         %\si=\sigma
\newcommand{\Si}{\Sigma}                         %\si=\sigma
\newcommand{\om}{\omega}                           %\ro=\rho
\newcommand{\Om}{\Omega}                           %\ta=tau





%Short for equation array and for equations
%%%%%%%%%%%%%%%%%%%%%%%%%%%%%%%%%%%%
\newcommand{\be}{\begin{equation}}               %\be=\begin{equation}
\newcommand{\ee}{\end{equation}}                 %\ee=\end{equation}
%%%%%%%%%%%%%%%%%%%%%%%%%%%%%%%%%%%%
\newcommand{\bea}{\begin{eqnarray}}              %\bea=\begin{eqnarray}
\newcommand{\eea}{\end{eqnarray}}                %\eea=\end{eqnarray}
%%%%%%%%%%%%%%%%%%%%%%%%%%%%%%%%%%%%
\newcommand{\bean}{\begin{eqnarray*}}            %\beq=\begin{eqnarray*}
\newcommand{\eean}{\end{eqnarray*}}              %\eeq=\end{eqnarray*}
%%%%%%%%%%%%%%%%%%%%%%%%%%%%%%%%%%%%
%\newcommand{\beq}{\begin{eqnarray*}}            %\beq=\begin{eqnarray*}
%\newcommand{\eeq}{\end{eqnarray*}}              %\eeq=\end{eqnarray*}
%%%%%%%%%%%%%%%%%%%%%%%%%%%%%%%%%%%%
\newcommand{\ba}{\begin{array}}                  %\ba=\begin{array}
\newcommand{\ea}{\end{array}}                    %\ea=\end{array}
%%%%%%%%%%%%%%%%%%%%%%%%%%%%%%%%%%%%
\newcommand{\nn}{\nonumber}                      %\nn=\nonumber
\newcommand{\mb}{\mbox}                          %\mb=\mbox





%\newcommand{\ra}{\rightarrow}                    %\ra=\rightarrow
\newcommand{\Ra}{\Rightarrow}                    %\ra=\rightarrow
\newcommand{\ua}{\uparrow}   
\newcommand{\da}{\downarrow}   
\newcommand{\Lra}{\Leftrightarrow}          %\llra=\longleftrightarrow
\newcommand{\llra}{\longleftrightarrow}          %\llra=\longleftrightarrow

\newcommand{\stac}{\stackrel}                    %\stac=\stackrel
\newcommand{\noin}{\noindent}                    %\noin=\noindent

%Symbol for the end of the proof
\newcommand{\qed}{\nobreak\quad\vrule width6pt depth3pt height10pt}

\newcommand{\ngi}{n \ra \infty}

%heading
%\pagestyle{myheadings} \markright{Rooted edges of a minimal directed spanning tree on random points}
\pagestyle{myheadings} \markright{[Homework] IID coin flips}

\thispagestyle{plain}


\begin{center}
{\Large\bf [Homework] IID coin flips} 
\end{center}






\begin{enumerate}



\item
The joint PMF of $X$ and $Y$ are given by

\begin{table}[H]    \center
\begin{tabular}{|c||c|c|c||c|} \hline
{\color{red}{\bf$y_j$}}&&&&                                     \\\hline\hline  
{\color{red}{\bf 3}}&$\f{2}{10}$&$\f{1}{10}$&$\f{1}{10}$&\\\hline
{\color{red}{\bf 2}}&$\f{1}{10}$&0&$\f{1}{10}$&\\\hline
{\color{red}{\bf 1}}&$\f{2}{10}$&$\f{1}{10}$&0&\\\hline
{\color{red}{\bf 0}}&0&$\f{1}{10}$&0&\\\hline\hline
&{\color{red}{\bf 0}}&{\color{red}{\bf 1}}&{\color{red}{\bf 2}}&{\color{red}{\bf$x_i$}}\\\hline
\end{tabular}
\end{table}

\begin{enumerate}
\item
Find the marginal PMF of $X$.
\\
{\color{blue}{\bf Sol.}}

\begin{table}[H]    \center
\begin{tabular}{|c||c|c|c||c|} \hline
           $i$        &   $0$      &   $1$   &$2$    & \\
\hline 
$P(X=i)$    & $\f{5}{10}$  & $\f{3}{10}$&$\f{1}{10}$ & 1\\
\hline


\end{tabular}\vspace{3mm}
%\caption{\small Bermudan Put option}
%\label{Vol10}
\end{table}


\item
Find the marginal PMF of $Y$.
\\
{\color{blue}{\bf Sol.}}
\begin{table}[H]    \center
\begin{tabular}{|c||c|c|c|c||c|} \hline
           $i$        &   $0$      &   $1$   &$2$    &$3$    &  \\
\hline 
$P(Y=i)$    & $\f{1}{10}$  & $\f{3}{10}$&$\f{2}{10}$ & $\f{4}{10}$ &1\\
\hline


\end{tabular}\vspace{3mm}
%\caption{\small Bermudan Put option}
%\label{Vol10}
\end{table}

\item
Find the conditional PMF of $X$ given $Y=1$.
\\
{\color{blue}{\bf Sol.}}
$$P(X=0|Y=1)=\f{P(x=0,Y=1)}{P(Y=1)}=\f{\f{2}{10}}{\f{3}{10}}=\f{2}{3}$$
$$P(X=1|Y=1)=\f{P(x=1,Y=1)}{P(Y=1)}=\f{\f{1}{10}}{\f{3}{10}}=\f{1}{3}$$
$$P(X=2|Y=1)=\f{P(x=2,Y=1)}{P(Y=1)}=\f{0}{\f{3}{10}}=0$$
\item
Find the conditional PMF of $Y$ given $X=2$.
\\
{\color{blue}{\bf Sol.}}
$$P(Y=0|X=2)=\f{P(x=2,Y=0)}{P(X=2)}=\f{0}{\f{2}{10}}=0$$
$$P(Y=1|X=2)=\f{P(x=2,Y=1)}{P(X=2)}=\f{0}{\f{2}{10}}=0$$
$$P(Y=2|X=2)=\f{P(x=2,Y=2)}{P(X=2)}=\f{\f{1}{10}}{\f{2}{10}}=\f{1}{2}$$
$$P(Y=3|X=2)=\f{P(x=2,Y=3)}{P(X=2)}=\f{\f{1}{10}}{\f{2}{10}}=\f{1}{2}$$
\end{enumerate}





\item 
The CDF $F$  is given by
$$
F(x)=\left\{\ba{ll}
0&\mb{for}\ x< 0\\
0.2&\mb{for}\ 0\le x<1\\
0.5&\mb{for}\ 1\le x<2\\
0.9&\mb{for}\ 2\le x<3\\
1&\mb{for}\ 3\le x
\ea\right.
$$
Compute the corresponding PMF, i.e.,
calculate $p_i=P(X=i)$, $i=0,1,2,3$.
\\
{\color{blue}{\bf Sol.}}


\begin{table}[H]    \center
\begin{tabular}{|c||c|c|c|c||c|} \hline
           $i$        &   $0$      &   $1$   &$2$    &$3$    &  \\
\hline 
$P(X=i)$    & $0.2$  & $0.3$&$0.4$ & $0.1$ &1\\
\hline


\end{tabular}\vspace{3mm}
%\caption{\small Bermudan Put option}
%\label{Vol10}
\end{table}













\item
Two balls are chosen randomly from an urn containing 8 white, 4 black, and 2 orange balls. 
Suppose that we win \$2 for each black ball selected and we lose \$1 for each white ball selected. 
Let $X$ denote our winnings. 
\begin{enumerate}
\item What are the possible values of $X$?
\\
{\color{blue}{\bf Sol.}}
The possible results of the experiment are \\
$$\{W,W \}, \{W,B \}, \{W,O \}, \{B,B \}, \{B,O \}, \{O,O \}$$
Then $X$ can take the values:\\
$-2$ for $\{W,W \}$, $1$ for $\{W,B \}$\\
$-1$ for $\{W,O \}$, $4$ for $\{B,B \}$\\
$2$ for $\{B,O \}$, $0$ for $\{O,O \}$\\
\item What are the probabilities associated with each value?
\\
{\color{blue}{\bf Sol.}}f
$$P(X=-2)=P(\{W,W\})=\f{{8 \choose 2}}{{14 \choose 2}}=\f{28}{91}$$
$$P(X=1)=P(\{W,B\})=\f{{8 \choose 1}\cdot {4 \choose 1}}{{14 \choose 2}}=\f{32}{91}$$
$$P(X=-1)=P(\{W,O\})=\f{{8 \choose 1}\cdot {2 \choose 1}}{{14 \choose 2}}=\f{16}{91}$$
$$P(X=4)=P(\{B,B\})=\f{{4 \choose 2}}{{14 \choose 2}}=\f{6}{91}$$
$$P(X=2)=P(\{B,O\})=\f{{4 \choose 1}\cdot {2 \choose 1}}{{14 \choose 2}}=\f{8}{91}$$
$$P(X=0)=P(\{O,O\})=\f{{2 \choose 2}}{{14 \choose 2}}=\f{1}{91}$$



\begin{table}[H]    \center
\begin{tabular}{|c||c|c|c|c|c|c||c|} \hline
                   & $-2$                & $-1$   &  $0$      &   $1$   &$2$   & $4$  & \\
\hline 
$p$    & $\f{28}{91}$  & $\f{16}{91}$&$\f{1}{91}$&$\f{32}{91}$    &   $\f{8}{91}$    &$\f{6}{91}$  &1\\
\hline



\end{tabular}\vspace{3mm}
%\caption{\small Bermudan Put option}
%\label{Vol10}
\end{table}

\end{enumerate}








\item  
A salesman has scheduled two appointments to sell encyclopedias. 
His first appointment will lead to a sale with probability 0.3, and 
his second will lead independently to a sale with probability 0.6. 
Any sale made is equally likely to be 
either for the deluxe model, which costs \$1000, 
or the standard model, which costs \$500. 
Determine the probability mass function of $X$, 
the total dollar value of all sales.
\\
{\color{blue}{\bf Sol.}}
For $i = 1, 2$ consider the events $S_i$ := $\{$sale on the $i$th appointment$\}$. We know that $S_1$ and $S_2$ are independent, $P(S_1)=0.3,P(S_2)=0.6$. Let $D_i$ := deluxe on $i$th, also. We know that $P(D_i|S_i)=P(D_i^c|S_i)=\f{1}{2}$. \\
Consequently, $P(S_i \cap D_i)=\f{P(S_i)}{2}$ and $P(S_i \cap D_i^c)=\f{P(S_i)}{2}$.\\
The possible values of $X$ are:\\
$2000$ dollars. In this case, we have\\ 
$$P\{X=2000\}=P(S_1 \cap D_1)P(S_2 \cap D_2)=\f{0.3}{2}\cdot\f{0.6}{2}=0.045$$\\
$1500$ dollars. In this case, we have\\ 
$$P\{X=1500\}=P(S_1 \cap D_1)P(S_2 \cap D_2^c)+P(S_1 \cap D_1^c)P(S_2 \cap D_2)=0.09$$\\
$1000$ dollars. In this case, we have \\
$$P\{X=1000\}=P(S_1 \cap D_1)P(S_2^c)+P(S_1^c)P(S_2 \cap D_2)+P(S_1 \cap D_1^c)P(S_2 \cap D_2^c)=0.315$$ 
$500$ dollars. In this case, we have \\
$$P\{X=500\}=P(S_1 \cap D_1^c)P(S_2^c)+P(S_1^c)P(S_2 \cap D_2^c)=0.27$$ 
$0$ dollars. In this case, we have \\
$$P\{X=0\}=P(S_1^c)P(S_2^c)=0.28$$ 


\item
Five distinct numbers are randomly distributed 
to players numbered 1 through 5. 
Whenever two players compare their numbers, 
the one with the higher one is declared the winner. 
Initially, players 1 and 2 compare their numbers; 
the winner then compares her number with that of player 3, and so on. 
Let $X$ denote the number of times player 1 is a winner. 
Find $P(X = i)$, $i = 0, 1, 2, 3, 4$.
\\
{\color{blue}{\bf Sol.}}
Let $Y_j$ denote the number distributed to player $j$. Note that $(Y_1 ,\cdots, Y_5)$ is a random permutation of $(1 ,\cdots, 5)$, all permutations being equally likely. Therefore, $p(0)=P\{X=0\}=P(Y_1<Y_2)$. But half of all permutations of $(1 ,\cdots, 5)$
have $Y_1 < Y_2$, whereas half have $Y_1 > Y_2$. Therefore,
$$p(0)=\f{\f{1}{2}\cdot5!}{5!}=\f{1}{2}$$
Next note that $p(1) = P\{Y_2 < Y_1 < Y_3\}$. The number of ways to end up with
$Y_2 < Y_1 < Y_3$ is the same as the number of ways to get $Y_1 < Y_2 < Y_3$. This is the same as $\cdots$. Therefore, the number of ways to get $Y_2 < Y_1 < Y_3$ is $\f{1}{3!}$ times the total number of permutations. That is,
$$p(1)=\f{\f{1}{3!}\cdot5!}{5!}=\f{1}{6}$$
\bea
p(2)
&=&P\{Y_2 < Y_1 , Y_3 < Y_1 , Y_4 > Y_1\}\nn\\
&=& P\{Y_1 = 3 , Y_2 = 1 , Y_3 = 2\} + P\{Y_1 = 3 , Y_2 = 2 , Y_3 = 1\}
\nn\\
&&+ P\{Y_1 = 4 , Y_2 = 1 , Y_3 = 2 , Y_4 = 5\} + P\{Y_1 = 4 , Y_2 = 2 , Y_3 = 1 , Y_4 = 5\}\nn\\
&&+ P\{Y_1 = 4 , Y_2 = 1 , Y_3 = 3 , Y_4 = 5\} + \cdots \nn\\
&=&(\f{2}{5!}+\f{2}{5!})+(\f{1}{5!}+\f{1}{5!}+\cdots)\nn\\
&=&(2\cdot\f{2}{5!})+(6\cdot\f{1}{5!})=\f{1}{12}\nn
\eea

Next we note that
\bea
p(3)
&=&P\{Y_2 < Y_1 , Y_3 < Y_1 , Y_4 < Y_1 , Y_5 > Y_1\}\nn\\
&=& P\{Y_1 = 4 , Y_1 = 1 , Y_2 = 2 , Y_3 = 3 , Y_4 = 4 , Y_5 = 5\}+\cdots \nn\\
&=&\f{3!}{5!}=\f{1}{20} \nn
\eea

Finally,
$$p(4) = P\{Y_1 = 5\} = \f{1}{5}$$



\item 
On a multiple-choice exam of 20 questions  with 4 possible answers for each question, 
let $S$ be the number of correct answers obtained just by guessing and
let $X_i$, $1\le i\le 20$, be the indicator of choosing the correct answer for the problem $i$.
\begin{enumerate}
\item Represent $S$ in terms of $X_i$.
\\
{\color{blue}{\bf Sol.}}
$$S=X_1+X_2+\cdots+X_{20}=\sum_{i=1}^{20}X_i$$
\item What is the distribution of $X_i$?
\\
{\color{blue}{\bf Sol.}}
$X_i$:Bernoulli distribution with $p=\f{1}{4}$. $X_i \sim B(\f{1}{4})$
\item Discuss the independence, pairwise independence, or dependence of $X_i$.
\\
{\color{blue}{\bf Sol.}}
Since
$$P(X_i=1,X_j=1)=P(X_i=1)P(X_j=1)$$
$$P(X_i=1,X_j=0)=P(X_i=1)P(X_j=0)$$
$$P(X_i=0,X_j=1)=P(X_i=0)P(X_j=1)$$
$$P(X_i=0,X_j=0)=P(X_i=0)P(X_j=0)$$
So, pairwise independence.\\
Since
$$P(X_1=a_1,X_2=a_2,\cdots X_{20}=a_{20})=P(X_1=a_1)P(X_2=a_2)\cdots P(X_{20}=a_{20})$$
where $a_1, \cdots a_{20}=0$ or $1$\\
independet.
\item Can we declare the distribution of $S$ is binomial? Why or why not.
\\
{\color{blue}{\bf Sol.}}
$$S=X_1+X_2+\cdots+X_{20}=\sum_{i=1}^{20}X_i  \sim  B(20,\f{1}{4}) $$
\end{enumerate}





\item 
We flip the fair coin 5 times independently and let $X$ be the number of heads.
Let $D$ be the number of heads minus the number of tails.
\begin{enumerate}
\item Represent $D$ in terms of $X$.
\\
{\color{blue}{\bf Sol.}}
$$D=2X-5 $$
\item What is the distribution of $X$?
\\
{\color{blue}{\bf Sol.}}
$$
X_i=\left\{\ba{ll}
1 &\mb{head}\ \\
0 &\mb{tail}\
\ea\right.
$$

$$X=X_1+X_2+\cdots +X_5$$
$$X_i \sim B(\f{1}{2}), X \sim B(5, \f{1}{2}) $$
\item Calculate the PMF of $D$.
\\
{\color{blue}{\bf Sol.}}
$$P(X=k)= P(D=2k-5)={5 \choose k}(\f{1}{2})^k(\f{1}{2})^{5-k},\ (k=0,1,\cdots ,4,5)$$

\begin{table}[H]    \center
\begin{tabular}{|c||c|c|c|c|c|c||c|} \hline
           $d$        &   $-5$      &   $-3$   &$-1$    & $1$& $3$ & $5$ & \\
\hline 
$P(D=d)$    & $ {5 \choose 0}(\f{1}{2})^5 $  & $ {5 \choose 1}(\f{1}{2})^5 $&${5 \choose 2}(\f{1}{2})^5 $ &$ {5 \choose 3}(\f{1}{2})^5 $ & $ {5 \choose 4}(\f{1}{2})^5 $ & $ {5 \choose 5}(\f{1}{2})^5 $ &  1\\
\hline


\end{tabular}\vspace{3mm}
%\caption{\small Bermudan Put option}
%\label{Vol10}
\end{table}

\end{enumerate}



\item 
We flip coins $n$ times independently and let $X$ be the number of heads.
For the $i$-th flip we use the $p_i$-coin which lands on head with probability $p_i$.
Let $A_i$ be the event that the $i$-coin lands on head
and let $1_{A_i}$ be its indicator.
\begin{enumerate}
\item Represent $X$ in terms of $1_{A_i}$.
\\
{\color{blue}{\bf Sol.}}
$$
1_{A_i}=\left\{\ba{ll}
1 &i\mb{th head}\ p_i\ \\
0 &\ $ {} $   \mb{tail}\ 1-p_i\
\ea\right.
$$
$$S=1_{A_1}+1_{A_2}+\cdots +1_{A_n}$$
\item What is the distribution of $1_{A_i}$?
\\
{\color{blue}{\bf Sol.}}
$$1_{A_i} \sim B(p_i)$$
\item Discuss the independence, pairwise independence, or dependence of $1_{A_i}$.
\\
{\color{blue}{\bf Sol.}}
$$P(1_{A_1} \cdots 1_{A_n})=P(1_{A_1})P(1_{A_2})\cdots P(1_{A_n})$$
$1_{A_i}$ is pairwise independence and independent.
\item Is the distribution of $X$ is binomial? Why or why not.
\\
{\color{blue}{\bf Sol.}}
If all $p_i$'s are same then $S \sim B(n,p)$.\\
If all $p_i$'s are diffenent then $S$ is not bibomial.\\
$1_{A_i}$:Bernolli distribution, independent\\
But, $S=\sum_{i=1}^n 1_{A_i}$ : not binomial. 
\end{enumerate}




















\item
A fair coin is tossed thirteen times, independently. Let $X$ be the number of times the coin lands heads in the first ten tosses, and let $Y$ be the number of times the coin lands tails in the last ten tosses.
Are $X$ and $Y$ independent?
\begin{enumerate}
\item Provide an intuitive argument.
\\
{\color{blue}{\bf Sol.}}
not independent.
$$P(X=5)={10 \choose 5}(\f{1}{2})^{10}$$
$$P(Y=7)={10 \choose 7}(\f{1}{2})^{10}$$
$$P(X=3,Y=10)=\f{1}{2^{13}}\cdot\{{7 \choose 4}\cdot{3 \choose 1}+{3 \choose 2}\cdot{7 \choose 5} \}$$
$$P(X=5)P(Y=7) \neq P(X=3,Y=10)$$
\item Provide a mathematical back up.
\\
{\color{blue}{\bf Sol.}}
$$P(X=3)={10 \choose 3}(\f{1}{2})^{3}, P(Y=10)={10 \choose 10}(\f{1}{2})^{10}$$
$$P(X=3,Y=10)=(\f{1}{2})^{13}$$
But,
$$P(X=3)\cdot  P(Y=10) \neq P(X=3,Y=10)$$
Therefore $X$ and $Y$ are not independent
\end{enumerate}







\item 
Initially a jar contains $n$ red candies and no white candies. 
A boy successively takes a candy from the jar and put a new white candy.
The boy draws a candy until he has a white candy. 
Let $X$ be the number of draws.
\begin{enumerate}
\item Can we declare the distribution of $X$ is geometric? Why or why not. 
\\
{\color{blue}{\bf Sol.}}
$X$ is not geometric because it is neither independent and identical
\item Compute the PMF of $X$.
\\
{\color{blue}{\bf Sol.}}
Let
$E_i$ = the event of not drawing a previous chip on the $i$th draw.\\
$p_i$ = probability of stopping on the $i$th draw\\
$n$ = the number of chips\\
$X$ = the number of draws before ending\\
Then,
\bea
P(X=x)
&=&P( \mb{probability of not stopping on any previous draws and stopping on the} x \mb{th draw}) \nn\\
&=& P(E_1E_2\cdots E_{x-1}(E_x)^c) \nn\\
&=&P(E_x^c | E_{x-1}E_{x-2}\cdots E_1)P(E_{x-1}  | E_{x-2}E_{x-3}\cdots E_1)\cdots P(E_2 | E_1)P(E_1) \nn
\eea
Now the probability of not drawing a previously seen chip given that there have been no repeat draws is just the number of unseen chips divided by the total number of chips. On draw $i$, this is just $\f{n-i+1}{n}$.\\ 
Similarly, the probability of drawing a previously seen chip on draw $x$, given that there have been no repeats thus far, is just $\f{x-1}{n}$. Thus the above becomes:
$$P(X=x)=\f{x-1}{n}\cdot \prod_{i=1}^{x-1} \f{n-i+1}{n}=\f{x-1}{n}\cdot\f{n!}{(n-x+1)!\cdot n^{x-1}}$$
Finally, since it is impossible to stop on the first draw, or draw a new one on the $(n+1)$ draw, we can give the following formula that includes the domain information:
$$
P(X=x)=\left\{\ba{ll}
\f{x-1}{n}\cdot \prod_{i=1}^{x-1} \f{n-i+1}{n} & x \in \{2,3,\cdots,n+1\}\ \\
0 &\mb{otherwise}\
\ea\right.
$$





\end{enumerate}





\end{enumerate}






\begin{center}
{\Large\bf [Extra] IID coin flips} 
\end{center}






\begin{enumerate}



\item 
Four buses carrying 148 students from the same school 
arrive at a football stadium. 
The buses carry, respectively, 40, 33, 25, and 50 students. 
One of the students is randomly selected. 
Let $X$ denote the number of students 
that were on the bus carrying the randomly selected student. 
One of the 4 bus drivers is also randomly selected. 
Let $Y$ denote the number of students on her bus.
Determine the PMFs of $X$ and $Y$.






\end{enumerate}


\end{document}
