\documentclass[12pt]{article}%report, article
%\documentstyle[12pt,leqno]{article}

\textwidth=15.5cm \textheight=21.6cm \topmargin=-0.5cm
\oddsidemargin=0.05cm

\newcommand{\bbA}{{\bf A}}
\newcommand{\bba}{{\bf a}}
\newcommand{\bbB}{{\bf B}}
\newcommand{\bbb}{{\bf b}}
\newcommand{\bbC}{{\bf C}}
\newcommand{\bbc}{{\bf c}}
\newcommand{\bbD}{{\bf D}}
\newcommand{\bbd}{{\bf d}}
\newcommand{\bbE}{{\bf E}}
\newcommand{\bbe}{{\bf e}}
\newcommand{\bbI}{{\bf I}}
\newcommand{\bbi}{{\bf i}}
\newcommand{\bbJ}{{\bf J}}
\newcommand{\bbj}{{\bf j}}
\newcommand{\bbK}{{\bf K}}
\newcommand{\bbk}{{\bf k}}
\newcommand{\bbP}{{\bf P}}
\newcommand{\bbp}{{\bf p}}
\newcommand{\bbQ}{{\bf Q}}
\newcommand{\bbq}{{\bf q}}
\newcommand{\bbT}{{\bf T}}
\newcommand{\bbt}{{\bf t}}
\newcommand{\bbU}{{\bf U}}
\newcommand{\bbu}{{\bf u}}
\newcommand{\bbV}{{\bf V}}
\newcommand{\bbv}{{\bf v}}
\newcommand{\bbW}{{\bf W}}
\newcommand{\bbw}{{\bf w}}
\newcommand{\bbX}{{\bf X}}
\newcommand{\bbx}{{\bf x}}
\newcommand{\X}{{\cal X}}
\newcommand{\bbY}{{\bf Y}}
\newcommand{\bby}{{\bf y}}
\newcommand{\bbZ}{{\bf Z}}
\newcommand{\bbz}{{\bf z}}
\newcommand{\0}{{\bf 0}}
\newcommand{\R}{{\bf R}}
\newcommand{\txi}{\bar{\xi}}
\def\Comment#1{ \marginpar{$\bullet$\quad{\tiny #1}}}


\usepackage{graphics,graphicx,amsmath,float,color,subfigure,enumerate,booktabs}
%\usepackage[tiling]{pst-fill}
\usepackage[dvips]{xy}
\usepackage{tikz}
\usetikzlibrary{matrix}
\input{rgb}
\xyoption{all}


\newcommand{\ra}[1]{\renewcommand{\arraystretch}{#1}}


\begin{document}
%controls the margin
\baselineskip=6.0mm








%Ignore some parts of statements
\newcommand{\ignore}[1]{}{}





%Equation numbers contain section number
\renewcommand{\theequation}{\arabic{section}.\arabic{equation}}





%Activate this if I want to display eq number only
\newcommand{\lbl}{\label}

%Activate this if I want to display eq number and text number, too
%\newcommand{\lbl}[1]{\hspace{1cm} \underline{({#1})} \label{#1}}





%Call this eq numbers in text
\newcommand{\eq}[1]{$(\ref{#1})$}


\newcommand{\f}{\frac}    


%Short for Greek letters
\newcommand{\al}{\alpha}                         %\al=\al
\newcommand{\bt}{\beta}                          %\bt=w
\newcommand{\ga}{\gamma}                         %\ga=\gamma
\newcommand{\Ga}{\Gamma}                         %\Ga=\Gamma
\newcommand{\de}{\delta}                         %\de=\delta
\newcommand{\De}{\Delta}                         %\De=\Delta
\newcommand{\ep}{\epsilon}                       %\ep=\epilon
\newcommand{\ve}{\varepsilon}                    %\ve=\varepsilon
\newcommand{\la}{\lambda}                        %\la=\lambda
\newcommand{\La}{\Lambda}                        %\La=\Lambda
\newcommand{\ro}{\rho}                           %\ro=\rho
\newcommand{\ta}{\tau}                           %\ta=tau
%\newcommand{\th}{\theta}                         %\th=\theta
\newcommand{\si}{\sigma}                         %\si=\sigma
\newcommand{\Si}{\Sigma}                         %\si=\sigma
\newcommand{\om}{\omega}                           %\ro=\rho
\newcommand{\Om}{\Omega}                           %\ta=tau





%Short for equation array and for equations
%%%%%%%%%%%%%%%%%%%%%%%%%%%%%%%%%%%%
\newcommand{\be}{\begin{equation}}               %\be=\begin{equation}
\newcommand{\ee}{\end{equation}}                 %\ee=\end{equation}
%%%%%%%%%%%%%%%%%%%%%%%%%%%%%%%%%%%%
\newcommand{\bea}{\begin{eqnarray}}              %\bea=\begin{eqnarray}
\newcommand{\eea}{\end{eqnarray}}                %\eea=\end{eqnarray}
%%%%%%%%%%%%%%%%%%%%%%%%%%%%%%%%%%%%
\newcommand{\bean}{\begin{eqnarray*}}            %\beq=\begin{eqnarray*}
\newcommand{\eean}{\end{eqnarray*}}              %\eeq=\end{eqnarray*}
%%%%%%%%%%%%%%%%%%%%%%%%%%%%%%%%%%%%
%\newcommand{\beq}{\begin{eqnarray*}}            %\beq=\begin{eqnarray*}
%\newcommand{\eeq}{\end{eqnarray*}}              %\eeq=\end{eqnarray*}
%%%%%%%%%%%%%%%%%%%%%%%%%%%%%%%%%%%%
\newcommand{\ba}{\begin{array}}                  %\ba=\begin{array}
\newcommand{\ea}{\end{array}}                    %\ea=\end{array}
%%%%%%%%%%%%%%%%%%%%%%%%%%%%%%%%%%%%
\newcommand{\nn}{\nonumber}                      %\nn=\nonumber
\newcommand{\mb}{\mbox}                          %\mb=\mbox





%\newcommand{\ra}{\rightarrow}                    %\ra=\rightarrow
\newcommand{\Ra}{\Rightarrow}                    %\ra=\rightarrow
\newcommand{\ua}{\uparrow}   
\newcommand{\da}{\downarrow}   
\newcommand{\Lra}{\Leftrightarrow}          %\llra=\longleftrightarrow
\newcommand{\llra}{\longleftrightarrow}          %\llra=\longleftrightarrow

\newcommand{\stac}{\stackrel}                    %\stac=\stackrel
\newcommand{\noin}{\noindent}                    %\noin=\noindent

%Symbol for the end of the proof
\newcommand{\qed}{\nobreak\quad\vrule width6pt depth3pt height10pt}

\newcommand{\ngi}{n \ra \infty}

%heading
%\pagestyle{myheadings} \markright{Rooted edges of a minimal directed spanning tree on random points}
\pagestyle{myheadings} \markright{[Homework] Detailed properties of probability}

\thispagestyle{plain}


\begin{center}
{\Large\bf [Homework] Detailed properties of probability} 
\end{center}






\begin{enumerate}





\item
We flip a fair coin many times.
Find the probability that
\begin{enumerate}
\item we have exactly 5 heads in the first 11 flips.
\\
{\color{blue}{\bf Sol.}}
${11 \choose 5}(\f{1}{2})^5(\f{1}{2})^6$
\item we have the very first head at the 11-th flip.
\\
{\color{blue}{\bf Sol.}}
$\f{1}{2^{11}}$
\item we have the very second head at the 11-th flip.
\\
{\color{blue}{\bf Sol.}}
${10 \choose 1}(\f{1}{2})^1\cdot (\f{1}{2})^9\cdot \f{1}{2}$

\item we have the very third head at the 11-th flip.
\\
{\color{blue}{\bf Sol.}}
${10 \choose 2}(\f{1}{2})^2\cdot (\f{1}{2})^8\cdot \f{1}{2}$
\item we have more heads than tails for the first time at the 11-th flip.
\\
{\color{blue}{\bf Sol.}}
$\f{{8 \choose 4}-{8 \choose 2}}{2^{11}}$



\end{enumerate}





\item 
A fair dice is rolled five times. 
What is the probability that
\begin{enumerate}
\item the second dice lands on a higher value than the first?
\\
{\color{blue}{\bf Sol.}}
$\f{15\cdot6^3}{6^5}$
\item 6 comes up at least once?
\\
{\color{blue}{\bf Sol.}}
(Complement)
Let $A$ be the event that there are at least one 6.
$A^c$ is the event that there are no 6 and 
$$
P(A^c)=(\f{5}{6})^5
\Ra
P(A)=1-P(A^c)=1-(\f{5}{6})^5
$$

\item all outcomes from the five rolls are identical?
\\
{\color{blue}{\bf Sol.}}
$\f{6}{6^5}$
\item the five outcomes are different?
\\
{\color{blue}{\bf Sol.}}
$\f{6\cdot5\cdot4\cdot3\cdot2}{6^5}$
\end{enumerate}




\item 
We have a colored fair dice. 
Each of the two faces of this dice are colored red, green, and blue.
We roll this fair dice 11 times. 
Find the probability that
\begin{enumerate}
\item we have exactly 5 reds in the first 11 flips.
\\
{\color{blue}{\bf Sol.}}
${11\choose 5}\cdot(\f{1}{3})^5\cdot(\f{2}{3})^6$
\item we have the very first red at the 11-th flip.
\\
{\color{blue}{\bf Sol.}}
${10\choose 0}\cdot(\f{1}{3})^0\cdot(\f{2}{3})^{10}\cdot(\f{1}{3})$
\item we have the very second red at the 11-th flip.
\\
{\color{blue}{\bf Sol.}}
${10\choose 1}\cdot(\f{1}{3})^1\cdot(\f{2}{3})^{9}\cdot(\f{1}{3})$
\item we have the very third red at the 11-th flip.
\\
{\color{blue}{\bf Sol.}}
${10\choose 2}\cdot(\f{1}{3})^2\cdot(\f{2}{3})^{8}\cdot(\f{1}{3})$
\end{enumerate}





\item
A roulette wheel has 38 spaces: 18 red, 18 black, and 2 green. 
The wheel will be spun five times.
Find the probability that
\begin{enumerate}
\item all the five land on red.
\\
{\color{blue}{\bf Sol.}}
${5\choose 5}\cdot(\f{18}{38})^5$
\item the first two land on red and the last three land on other colors.
\\
{\color{blue}{\bf Sol.}}
$(\f{18}{38})^2\cdot(\f{20}{38})^3$
\item two land on red and three land on other colors.
\\
{\color{blue}{\bf Sol.}}
${5\choose 2}\cdot(\f{18}{38})^2\cdot(\f{20}{38})^3$
\end{enumerate}





\item 
There are 10 hotels in a certain town. 
If 7 different group of people check into hotels in a day, 
what is the probability that each check into a different hotel? 
\\
{\color{blue}{\bf Sol.}}
When the first person checks into the hotel, the next person will check into a different hotel with probability $\f{9}{10}$. The next person will check into a different hotel with probability $\f{8}{10}$.Thus the probability that we check into seven different hotels is given by
$$\f{10}{10} \cdot \f{9}{10} \cdot \f{8}{10} \cdots \f{4}{10}=\f{{10 \choose 7}\cdot7!}{10^7}$$




\item
In a drawer there are 10 red, 3 green and 7 blue socks.
Seven socks are withdrawn. 
Find the probability that 
\begin{enumerate}
\item 3 red, 2 green, and 2 blue socks are withdrawn.
\\
{\color{blue}{\bf Sol.}}
$\f{{10 \choose 3}\cdot{3 \choose 2}\cdot{7 \choose 2}}{{20 \choose 7}}$  
\item at least 2 red socks are withdrawn.
\\
{\color{blue}{\bf Sol.}}
$1-P(\mb{no red socks withdrawn})-P(\mb{only 1 red socks withdrawn})$\\
=$1-\f{{10 \choose 0}\cdot{10 \choose 7}}{{20 \choose 7}}-\f{{10 \choose 1}\cdot{10 \choose 6}}{{20 \choose 7}}$
\item exactly one color is missing among the three withdrawn socks.
\\
{\color{blue}{\bf Sol.}}
$1-\f{{10 \choose 3}+{3 \choose 3}+{7 \choose 3}+{10 \choose 1}\cdot{3 \choose 1}\cdot{7 \choose 1}}{{20 \choose 3}}$
\item exactly two colors are missing among the three withdrawn socks.
\\
{\color{blue}{\bf Sol.}}
$\f{{10 \choose 3}+{3 \choose 3}+{7 \choose 3}}{{20 \choose 3}}$
\item no colors are missing among the three withdrawn socks.
\\
{\color{blue}{\bf Sol.}}
$\f{{10 \choose 1}\cdot{3 \choose 1}\cdot{7 \choose 1}}{{20 \choose 3}}$
\end{enumerate}











\item
In a drawer there are 10 red, 3 green and 7 blue socks.
Starting from Author,  Author and Bob withdraw socks  from the draw consecutively
until a green sock is selected. 
Find the probability that Author selects the green sock.
\\
{\color{blue}{\bf Sol.}}
$\f{3}{20}+\f{17}{20}\cdot\f{16}{19}\cdot\f{3}{18}+\f{17}{20}\cdot\f{16}{19}\cdot\f{15}{18}\cdot\f{14}{17}\cdot\f{3}{16}+\f{17}{20}\cdot\f{16}{19}\cdot\f{15}{18}\cdot\f{14}{17}\cdot\f{13}{16}\cdot\f{12}{15}\cdot\f{3}{14}+\f{17}{20}\cdot\f{16}{19}\cdot\f{15}{18}\cdot\f{14}{17}\cdot\f{13}{16}\cdot\f{12}{15}\cdot\f{11}{14}\cdot\f{10}{13}\cdot\f{3}{12}+\f{17}{20}\cdot\f{16}{19}\cdot\f{15}{18}\cdot\f{14}{17}\cdot\f{13}{16}\cdot\f{12}{15}\cdot\f{11}{14}\cdot\f{10}{13}\cdot\f{9}{12}\cdot\f{8}{11}\cdot\f{3}{10}+\f{17}{20}\cdot\f{16}{19}\cdot\f{15}{18}\cdot\f{14}{17}\cdot\f{13}{16}\cdot\f{12}{15}\cdot\f{11}{14}\cdot\f{10}{13}\cdot\f{9}{12}\cdot\f{8}{11}\cdot\f{7}{10}\cdot\f{6}{9}\cdot\f{3}{8}+\f{17}{20}\cdot\f{16}{19}\cdot\f{15}{18}\cdot\f{14}{17}\cdot\f{13}{16}\cdot\f{12}{15}\cdot\f{11}{14}\cdot\f{10}{13}\cdot\f{9}{12}\cdot\f{8}{11}\cdot\f{7}{10}\cdot\f{6}{9}\cdot\f{5}{8}\cdot\f{4}{7}\cdot\f{3}{6}+\f{17}{20}\cdot\f{16}{19}\cdot\f{15}{18}\cdot\f{14}{17}\cdot\f{13}{16}\cdot\f{12}{15}\cdot\f{11}{14}\cdot\f{10}{13}\cdot\f{9}{12}\cdot\f{8}{11}\cdot\f{7}{10}\cdot\f{6}{9}\cdot\f{5}{8}\cdot\f{4}{7}\cdot\f{3}{6}\cdot\f{2}{5}\cdot\f{3}{4}=\f{41}{76}$

\item
In a certain region there are many elks.
We captured, tagged, and then released  5 elks. 
A certain time later, we captured 4 elks and 2 of these 4 have been tagged. 
\begin{enumerate}
\item Compute the probability that this happens as a function of the total number $n$ of the elks in the region.
\\
{\color{blue}{\bf Sol.}}
The number of total events is ${n\choose 4}$. And there are ${5\choose 2}\cdot {n-5 \choose 2}$ different ways to choose 2 from the previously tagged elks, and 2 from the elks which have not been tagged before.
\\
Therefore, the probability is $\f{{5\choose 2}\cdot {n-5 \choose 2}}{{n\choose 4}}$
different ways to choose
2 from the previously tagged elks, and 2 from the elks which have not been tagged before.
\item Find $n$ which maximizes this probability.
\\
{\color{blue}{\bf Sol.}}
Let $f(n) = \f{{5\choose 2}\cdot {n-5 \choose 2}}{{n\choose 4}} = \f{120(n-5)(n-6)}{n(n-1)(n-2)(n-3)}$\\
We want to find $n$ such that 
$$\f{f(n)}{f(n-1)}\ge 1, \f{f(n+1)}{f(n)}\le 1$$
$$9 \le n \le 10 $$
So $n$ is 9 or 10, $f(9)=\f{10}{21}, f(10)=\f{10}{21}$

\end{enumerate}

















\item 
An instructor gives her class a set of 20 problems 
with the information that the final exam will consist of a random selection of 10 of them. 
If a student has figured out how to do 15 of the problems, 
what is the probability that he or she will answer correctly
\begin{enumerate}
\item all 10 problems?
\\
{\color{blue}{\bf Sol.}}
$\f{{15 \choose 10}}{{20 \choose 10}}$ 
\item at least 8 of the problems?
\\
{\color{blue}{\bf Sol.}}
$\f{{15 \choose 8}\cdot{5 \choose 2}}{{20 \choose 10}}
+ \f{{15 \choose 9}\cdot{5 \choose 1}}{{20 \choose 10}}
+\f{{15 \choose 10}\cdot{5 \choose 0}}{{20 \choose 10}}=\f{1}{2}$ 
\end{enumerate}










\item 
A closet contains 10 pairs of shoes. 
If 8 shoes are randomly selected, what is the probability that there will be
\begin{enumerate}
\item no complete pair?
\\
{\color{blue}{\bf Sol.}}
 10 pairs = 20 shoes
\\
There are ${20 \choose 8}$ ways to select the 8 shoes.
\\
There are ${10 \choose 8}$ ways to choose which pairs contribute one shoe each, and two possibilities for picking the shoe from each of the pairs (left or right).
\\
This gives a total of ${10 \choose 8}\cdot 2^8$ ways to choose 8 shoe that are unpaired
$$\f{{10 \choose 8}\cdot 2^8}{{20 \choose 8}} $$

  
      
\item exactly 1 complete pair?
\\
{\color{blue}{\bf Sol.}}
There are 10 ways to choose the single pair of shoes and 
 ${9 \choose 6}\cdot 2^6$ ways to choose the remaining six shoes so that they unpaired
$$\f{10\cdot{9 \choose 6}\cdot 2^6}{{20 \choose 8}} $$
\end{enumerate}








\end{enumerate}




\begin{center}
{\Large\bf [Extra] Detailed properties of probability} 
\end{center}






\begin{enumerate}


\item 
We choose balls successively without replacement 
from an urn containing 20 red  and 10 blue balls. 
What is the probability that all the red balls are removed before all the blue ones are removed from the urn? 


\end{enumerate}

\end{document}
