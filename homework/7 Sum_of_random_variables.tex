\documentclass[12pt]{article}%report, article
%\documentstyle[12pt,leqno]{article}

\textwidth=15.5cm \textheight=21.6cm \topmargin=-0.5cm
\oddsidemargin=0.05cm

\newcommand{\bbA}{{\bf A}}
\newcommand{\bba}{{\bf a}}
\newcommand{\bbB}{{\bf B}}
\newcommand{\bbb}{{\bf b}}
\newcommand{\bbC}{{\bf C}}
\newcommand{\bbc}{{\bf c}}
\newcommand{\bbD}{{\bf D}}
\newcommand{\bbd}{{\bf d}}
\newcommand{\bbE}{{\bf E}}
\newcommand{\bbe}{{\bf e}}
\newcommand{\bbI}{{\bf I}}
\newcommand{\bbi}{{\bf i}}
\newcommand{\bbJ}{{\bf J}}
\newcommand{\bbj}{{\bf j}}
\newcommand{\bbK}{{\bf K}}
\newcommand{\bbk}{{\bf k}}
\newcommand{\bbP}{{\bf P}}
\newcommand{\bbp}{{\bf p}}
\newcommand{\bbQ}{{\bf Q}}
\newcommand{\bbq}{{\bf q}}
\newcommand{\bbT}{{\bf T}}
\newcommand{\bbt}{{\bf t}}
\newcommand{\bbU}{{\bf U}}
\newcommand{\bbu}{{\bf u}}
\newcommand{\bbV}{{\bf V}}
\newcommand{\bbv}{{\bf v}}
\newcommand{\bbW}{{\bf W}}
\newcommand{\bbw}{{\bf w}}
\newcommand{\bbX}{{\bf X}}
\newcommand{\bbx}{{\bf x}}
\newcommand{\X}{{\cal X}}
\newcommand{\bbY}{{\bf Y}}
\newcommand{\bby}{{\bf y}}
\newcommand{\bbZ}{{\bf Z}}
\newcommand{\bbz}{{\bf z}}
\newcommand{\0}{{\bf 0}}
\newcommand{\R}{{\bf R}}
\newcommand{\txi}{\bar{\xi}}
\def\Comment#1{ \marginpar{$\bullet$\quad{\tiny #1}}}


\usepackage{graphics,graphicx,amsmath,float,color,subfigure,enumerate,booktabs}
%\usepackage[tiling]{pst-fill}
\usepackage[dvips]{xy}
\usepackage{tikz}
\usetikzlibrary{matrix}
\input{rgb}
\xyoption{all}


\newcommand{\ra}[1]{\renewcommand{\arraystretch}{#1}}


\begin{document}
%controls the margin
\baselineskip=6.0mm








%Ignore some parts of statements
\newcommand{\ignore}[1]{}{}





%Equation numbers contain section number
\renewcommand{\theequation}{\arabic{section}.\arabic{equation}}





%Activate this if I want to display eq number only
\newcommand{\lbl}{\label}

%Activate this if I want to display eq number and text number, too
%\newcommand{\lbl}[1]{\hspace{1cm} \underline{({#1})} \label{#1}}





%Call this eq numbers in text
\newcommand{\eq}[1]{$(\ref{#1})$}


\newcommand{\f}{\frac}    


%Short for Greek letters
\newcommand{\al}{\alpha}                         %\al=\al
\newcommand{\bt}{\beta}                          %\bt=w
\newcommand{\ga}{\gamma}                         %\ga=\gamma
\newcommand{\Ga}{\Gamma}                         %\Ga=\Gamma
\newcommand{\de}{\delta}                         %\de=\delta
\newcommand{\De}{\Delta}                         %\De=\Delta
\newcommand{\ep}{\epsilon}                       %\ep=\epilon
\newcommand{\ve}{\varepsilon}                    %\ve=\varepsilon
\newcommand{\la}{\lambda}                        %\la=\lambda
\newcommand{\La}{\Lambda}                        %\La=\Lambda
\newcommand{\ro}{\rho}                           %\ro=\rho
\newcommand{\ta}{\tau}                           %\ta=tau
%\newcommand{\th}{\theta}                         %\th=\theta
\newcommand{\si}{\sigma}                         %\si=\sigma
\newcommand{\Si}{\Sigma}                         %\si=\sigma
\newcommand{\om}{\omega}                           %\ro=\rho
\newcommand{\Om}{\Omega}                           %\ta=tau





%Short for equation array and for equations
%%%%%%%%%%%%%%%%%%%%%%%%%%%%%%%%%%%%
\newcommand{\be}{\begin{equation}}               %\be=\begin{equation}
\newcommand{\ee}{\end{equation}}                 %\ee=\end{equation}
%%%%%%%%%%%%%%%%%%%%%%%%%%%%%%%%%%%%
\newcommand{\bea}{\begin{eqnarray}}              %\bea=\begin{eqnarray}
\newcommand{\eea}{\end{eqnarray}}                %\eea=\end{eqnarray}
%%%%%%%%%%%%%%%%%%%%%%%%%%%%%%%%%%%%
\newcommand{\bean}{\begin{eqnarray*}}            %\beq=\begin{eqnarray*}
\newcommand{\eean}{\end{eqnarray*}}              %\eeq=\end{eqnarray*}
%%%%%%%%%%%%%%%%%%%%%%%%%%%%%%%%%%%%
%\newcommand{\beq}{\begin{eqnarray*}}            %\beq=\begin{eqnarray*}
%\newcommand{\eeq}{\end{eqnarray*}}              %\eeq=\end{eqnarray*}
%%%%%%%%%%%%%%%%%%%%%%%%%%%%%%%%%%%%
\newcommand{\ba}{\begin{array}}                  %\ba=\begin{array}
\newcommand{\ea}{\end{array}}                    %\ea=\end{array}
%%%%%%%%%%%%%%%%%%%%%%%%%%%%%%%%%%%%
\newcommand{\nn}{\nonumber}                      %\nn=\nonumber
\newcommand{\mb}{\mbox}                          %\mb=\mbox





%\newcommand{\ra}{\rightarrow}                    %\ra=\rightarrow
\newcommand{\Ra}{\Rightarrow}                    %\ra=\rightarrow
\newcommand{\ua}{\uparrow}   
\newcommand{\da}{\downarrow}   
\newcommand{\Lra}{\Leftrightarrow}          %\llra=\longleftrightarrow
\newcommand{\llra}{\longleftrightarrow}          %\llra=\longleftrightarrow

\newcommand{\stac}{\stackrel}                    %\stac=\stackrel
\newcommand{\noin}{\noindent}                    %\noin=\noindent

%Symbol for the end of the proof
\newcommand{\qed}{\nobreak\quad\vrule width6pt depth3pt height10pt}

\newcommand{\ngi}{n \ra \infty}

%heading
%\pagestyle{myheadings} \markright{Rooted edges of a minimal directed spanning tree on random points}
\pagestyle{myheadings} \markright{[Homework] Sum of random variables}

\thispagestyle{plain}


\begin{center}
{\Large{\bf [Homework] Sum of random variables}} 
\end{center}


\begin{enumerate}





 \item
We flips a $p$-coin many times, where $p=0.40$.
\bea
X_i&&\mb{$i^{th}$ flip record,
where $H$ and $T$ are recorded as 1 and 0}\nn\\
Y_i:=2X_i-1&&\mb{$i^{th}$ flip record,
where $H$ and $T$ are recorded as 1 and $-1$}\nn
\eea

Calculate the mean and variance of the following related random variables,
i.e., fill up blanks of the below table.
\begin{table}[H]    \center
\begin{tabular}{|c||c|c|c||c|} \hline
Random variable&Mean&Variance\\\hline\hline  
$Y_i$&$-0.2$&$0.96$\\\hline
$\sum_{i=1}^nY_i$&$-0.2n$&$0.96n$\\\hline
$\f{\sum_{i=1}^nY_i}{\sqrt{n}}$&$-0.2\sqrt{n}$&$0.96$\\\hline
\end{tabular}
\end{table}





 
 
 
 
 
 
\item
Suppose that $A$ and $B$ each randomly and independently choose 3 different integers from $\{1,2,\cdots,10\}$. Find the expected number of integers
\begin{enumerate}
\item chosen by both $A$ and $B$.
\\
{\color{blue}{\bf Sol.}}
Let $X$ be the number of objects that are selected by both $A$ and $B$. To further simplify
 the problem we use indicator variables $X_i$. Let $X_i = 1$ if object $i$ is selected by both $A$
 and $B$, and $X_i = 0$ otherwise, where $1\le i\le10$. Then,

$$E(X)=E(\sum_{i=1}^{10} X_i)=\sum_{i=1}^{10} E(X_i)$$

Now we must find $E(X_i)$. We know that $X_i$ only takes on one of two values, $X_i = 1$ or
 $X_i = 0$. So, for the case of a sum of independent random indicator variables, $E(X_i) =
 P(X_i = 1)$.
 Each person can choose 3 of the 10 items. There are 3 ways to choose the item of
 interest, since a person can draw 3 objects. Since person A and B draw independently,

$$P(X_i=1)=(\f{3}{10})^2$$

Then,

$$E(X)=\sum_{i=1}^{10} E(X_i)=\sum_{i=1}^{10} (\f{3}{10})^2 =\f{9}{10}$$




\item not chosen by either $A$  or $B$.
\\
{\color{blue}{\bf Sol.}}
The principle is similar to part (a). Let $X_i = 1$ if object $i$ is not chosen by $A$ and is not chosen by $B$. $P(X_i = 1) = (\f{7}{10})^2$, because the probability that an arbitrary person does
 not choose object $i$ is $\f{7}{10}$ and person A and person B draw independently. Then,

$$E(X)=\sum_{i=1}^{10} E(X_i)=\sum_{i=1}^{10} (\f{7}{10})^2 =4.9$$




\item chosen by exactly one of $A$ and $B$.
\\
{\color{blue}{\bf Sol.}}
In this case, either person $A$ draws object $i$ and person $B$ does not, or person $B$ draws object $i$ and person $A$ does not. Again, let $X_i = 1$ if exactly one of $A$ or $B$ draws object $i$, $X_i = 0$ otherwise. The person that eventually draws object $i$ had probability $\f{3}{10}$ of
 drawing the object and the person that does not draw object $i$ had probability $\f{7}
 {10}$ of not drawing object $i$. But there are two ways to arrange this situation. A can draw the
 object, and $B$ does not, or $B$ draws the object and $A$ does not. Thus,
$$E(X_i)=P(X_i=1)=2 \cdot \f{3}{10} \cdot \f{7}{10}$$
and
$$E(X)=10(2\cdot\f{3}{10}\cdot \f{7}{10})=4.2$$


\end{enumerate}
 
 
 
 
 




\item
A group of eighteen students contains seven business majors; the rest are all liberal arts majors. 
Five students will be drawn from the group and let $X$ be the number of business majors in the sample.
\begin{enumerate}
\item Calculate $P(X=3)$.
\\
{\color{blue}{\bf Sol.}}
$$P(X=3)=\f{{7 \choose 3}{11 \choose 2}}{{18 \choose 5}}$$
\item What is the distribution of $X$?
\\
{\color{blue}{\bf Sol.}}
$X \sim $ Hypergeometric distribution $H(5,7,18)$
\item What is $EX$?
\\
{\color{blue}{\bf Sol.}}
$$EX=n\cdot \f{m}{N}=5\cdot \f{7}{18}=\f{35}{18}$$
\item What is $Var(X)$?
\\
{\color{blue}{\bf Sol.}}
$$Var(X)=n\cdot\f{m}{N}(1-\f{m}{N})(1-\f{n-1}{N-1})=5\cdot\f{7}{18}(1-\f{7}{18})(1-\f{5-1}{18-1})$$
\end{enumerate}


\item
A pond contains 100 fish, of which 30 are carp. If 20 fish are caught, what are the mean and variance of the number of carp among the 20? 
\\
{\color{blue}{\bf Sol.}}
Let $X$ be the number of carp. Then $X \sim H(20,30,100)$ 
$$EX=n\cdot \f{m}{N}=20\cdot \f{30}{100}=6$$
$$Var(X)=n\cdot\f{m}{N}(1-\f{m}{N})(1-\f{n-1}{N-1})=20 \cdot\f{30}{100}(1-\f{30}{100})(1-\f{20-1}{100-1})$$





\item
For a group of 100 people, compute
\begin{enumerate}
\item the expected number of days of the year that
are birthdays of exactly 3 people.
\\
{\color{blue}{\bf Sol.}}
Let $X$ be the number of days of the year that are birthdays for exactly $3$ people. Let
 $X_i = 1$ if day $i$ is a birthday for $3$ people. This is a binomial problem with $n = 100$ and $p = (\f{1}{365})^3$. Then,
$$E(X_i) = P(X_i = 1) = { 100 \choose 3}(\f{1}{365})^3(\f{364}{365})^{97}$$
Recall that $i$ represents a day of the year, so in the following calculation, we sum $i = 1$ to $i = 365$.
$$E(X)=\sum_{i=1}^{365} E(X_i)=365\cdot{ 100 \choose 3}(\f{1}{365})^3(\f{364}{365})^{97}$$

\item the expected number of days of the year that
are birthdays of someone.
\\
{\color{blue}{\bf Sol.}}
You may be tempted to think of this problem as the number of people that have different
 birthdays, but that is incorrect. Let $X$ be the number of distinct birthdays. that is, the number of days of the year that are occupied by a birthday. Let $X_i = 1$ if someone
 has a birthday on day $i$. We use the complement. We find the probability that all $100$ people have different birthdays and subtract from $1$.
$$E(X_i)=1-(\f{364}{365})^{100}$$
Then,
$$E(X)=\sum_{i=1}^{365} E(X_i)=365\cdot \{1-(\f{364}{365})^{100} \}$$

\end{enumerate}
 
 




\item
We have 10 different points on the unit circle.
For any two point
we flip a fair coin independently and,
if the coin lands on head, we join these two by a line segment.
Calculate the mean and variance of the number of triangles formed. 
\\
{\color{blue}{\bf Sol.}}
$$
A_{i,j,k}=\left\{\ba{ll}
1 & i,j,k \  \mb{are connected} \ p=(\f{1}{2})^3 \\
0 &\mb{otherwise}\ q=1-(\f{1}{2})^3\
\ea\right.
$$
$$EX=(E \sum_{i,j,k} 1_{A_{i,j,k}}) = {10 \choose 3}(\f{1}{2})^3=15$$
We have to find $EX^2$


\begin{figure}[h] 
\centering
\includegraphics[width=6in,height=4in]{hw7.png}
\caption{$EX^2$}
\end{figure}



$$(\f{1}{2})^6{10 \choose 3}{7 \choose 3}=\f{525}{8}$$
$$3\cdot(\f{1}{2})^6{10 \choose 3}{7 \choose 2}=\f{945}{8}$$
$$3\cdot(\f{1}{2})^5{10 \choose 3}{7 \choose 1}=\f{630}{8}$$
$$(\f{1}{2})^3{10 \choose 3}=15$$

$$E(X^2)=\f{525}{8}+\f{945}{8}+\f{630}{8}+15=\f{2220}{8}$$
$$Var(X)=E(X^2)-(EX)^2=\f{105}{2}$$





\item
How many times would you expect to roll a fair die to see all 6 sides appeared at least once?
How about the variance?
\\
{\color{blue}{\bf Sol.}}
The time until the first result appears is $1$. After that, the random time until a second (different) result appears is geometrically distributed with parameter of success $\f{5}{6}$, hence with mean $\f{6}{5}$. After that, the random time until a third (different) result appears is geometrically distributed with parameter of success $\f{4}{6}$, hence with mean $\f{6}{4}$. And so on, until the random time of appearance of the last and sixth result, which is geometrically distributed with parameter of success $\f{1}{6}$, hence with mean $\f{6}{1}$. So the mean total time to get all six results is
$$E(X)=\f{6}{6}+\f{6}{5}+\f{6}{4}+\f{6}{3}+\f{6}{2}+\f{6}{1}=14.7$$ 
$$Var(X)=\f{0}{(\f{6}{6})^2}+\f{\f{1}{6}}{(\f{5}{6})^2}+\f{\f{2}{6}}{(\f{4}{6})^2}+\f{\f{3}{6}}{(\f{3}{6})^2}+\f{\f{4}{6}}{(\f{2}{6})^2}+\f{\f{5}{6}}{(\f{1}{6})^2}=\f{3899}{100}$$





 

\item
Consider $n$ independent flips of a fair coin. 
Say that a changeover occurs whenever an outcome differs from the one preceding it. For instance, if $n=12$ and the outcome is $HHTTTTHHHHHT$, then there are 3 changeovers.
$$
\ba{cccccccccccc}
H&H&T&T&T&T&H&H&H&H&H&T\\
&&\uparrow&&&&\uparrow&&&&&\uparrow
\ea
$$
$$
\ \ \ \ \ \ \ \ \ \ \mb{changeover}\ \ \ \ \ \mb{changeover}\ \ \ \ \ \ \ \ \ \ \ \ \mb{changeover}
$$ 
Let $X$ be the number of changeovers during the $n$ flips.
\begin{enumerate}
\item
Calculate $P(X=1)$. 
\\
{\color{blue}{\bf Sol.}}
For $i=2, \cdots , n$
$$
X_i=\left\{\ba{ll}
1 &\mb{if chageover occurs at} i\ p=\f{1}{2} \\
0 &\mb{otherwise}\ q=\f{1}{2}\
\ea\right.
$$
Then $X_i \sim B(\f{1}{2})$
$$X=X_2+X_3+\cdots+X_{n}$$
$$X\sim B(n-1,\f{1}{2})$$
$$P(X=1)={ n-1 \choose 1}(\f{1}{2})^1(\f{1}{2})^{n-2}=(n-1)(\f{1}{2})^{n-1}$$
\item
Calculate the mean and variance of $X$. 
\\
{\color{blue}{\bf Sol.}}
\bea
E(X)
&=&E(\sum_{i=2}^n X_i) \nn\\
&=&\sum_{i=2}^n E(X_i)= \sum_{i=2}^n 2p(1-p)\nn\\
&=&2(n-1)p(1-p)=\f{1}{2}(n-1) \nn
\eea


$$Var(X)=npq=(n-1)\cdot\f{1}{2}\cdot\f{1}{2}=\f{n-1}{4}$$
\end{enumerate}




\item
A box contains five tickets labeled with numbers $-3, -1, 0, 1, 3$. 
We do 100 random draws with replacement from the box.
Calculate the expectation and variance of the sum of the numbers on the tickets drawn.
\\
{\color{blue}{\bf Sol.}}
Each of the five numbers should be drawn about $\f{100}{5} = 20$ times. 
Let $X$ be the sum of the numbers on the tickets drawn.
$$EX=(-3)\cdot20+(-1)\cdot20+0\cdot20+1\cdot20+3\cdot20=0$$
$$VarX=(-3)^2\cdot20+(-1)^2\cdot20+0^2\cdot20+1^2\cdot20+3^2\cdot20-0^2=400$$

\item
Form 10 teams of 2 from 10 men and 10 women, randomly.
\begin{enumerate}
\item
Calculate the mean and variance of the number of teams of different sex. 
\\
{\color{blue}{\bf Sol.}}
Let $X_i$ be the $i$th team.
$$P(X_i=1)=\f{{10 \choose 1}\cdot {10 \choose 1}}{{20 \choose 2}}=\f{10}{19}$$
$$
X_i=\left\{\ba{ll}
1 &\ i\mb{th teams of different sex}, p=\f{10}{19}\\
0 &\ \mb{otherwise}\
\ea\right.
$$
$$E(1_{X_i})=\f{10}{19}$$
So
$$EX=E(\sum_{i=1}^{10} 1_{X_i})=10\cdot\f{10}{19}=\f{100}{19}$$
$$E(X_iX_j)=P(X_i=1)P(X_j=1|X_i=1)=\f{16}{19}\cdot\f{{18 \choose 1}\cdot {19 \choose 1}}{{18 \choose 2}}=\f{10}{19}\cdot\f{9}{17}$$

\bea
VarX
&=&\sum_{i=1}^{10} Var(X_i) + \sum_{i=1}^{10}\sum_{j=1,i\neq j}^{10}Cov(X_i,X_j) \nn\\
&=&\sum_{i=1}^{10} Var(X_i) + \sum_{i=1}^{10}\sum_{j=1,i\neq j}^{10}\{E(X_iX_j)-EX_iEX_j\}\nn\\
&=&10\cdot\f{10}{19}\cdot(1-\f{10}{19})+10\cdot9 \{ \f{10}{19}\cdot\f{9}{17}-(\f{10}{19})^2 \}\nn
\eea

\item
If 20 people are actually 10 couples,
calculate the mean and variance of the number of teams of couple. 
\\
{\color{blue}{\bf Sol.}}
Let $Y_i$ be the $i$th couple.
$$P(Y_i=1)=\f{1}{19}$$
$$
Y_i=\left\{\ba{ll}
1 &\ i\mb{th team is couple}, p=\f{1}{19}\\
0 &\ \mb{otherwise}\
\ea\right.
$$

$$EY=\sum_{i=1}^{10}E(1_{Y_i})=\f{10}{19}$$
$$E(1_{Y_i}1_{Y_j})=P(Y_i=1)P(Y_j=1|Y_i=1)=\f{1}{19}\cdot\f{1}{17}$$
\bea
VarY
&=&\sum_{i=1}^{10} Var(Y_i) + \sum_{i=1}^{10}\sum_{j=1,i\neq j}^{10}Cov(Y_i,Y_j) \nn\\
&=&10Var(Y_i) + 10\cdot9 \{E(Y_iY_j)-EY_iEY_j\}\nn\\
&=&10\cdot\f{1}{19}\cdot(1-\f{1}{19})+90\cdot \{ \f{1}{19}\cdot\f{1}{17}-(\f{1}{19})^2 \}\nn
\eea

\end{enumerate}


\end{enumerate}
\vskip 0.3in




\begin{center}
{\Large{\bf Extra}} 
\end{center}


\begin{enumerate}









\item
Initially 
100 black balls are in the bin. At each stage, you choose one ball from the bin, remove the chosen ball, and add one white ball.
Compute 
the expectation and variance of the number of stages needed until there are no more black balls in the bin.



\end{enumerate}



\end{document}
