\documentclass[12pt]{article}%report, article
%\documentstyle[12pt,leqno]{article}

\textwidth=15.5cm \textheight=21.6cm \topmargin=-0.5cm
\oddsidemargin=0.05cm

\newcommand{\bbA}{{\bf A}}
\newcommand{\bba}{{\bf a}}
\newcommand{\bbB}{{\bf B}}
\newcommand{\bbb}{{\bf b}}
\newcommand{\bbC}{{\bf C}}
\newcommand{\bbc}{{\bf c}}
\newcommand{\bbD}{{\bf D}}
\newcommand{\bbd}{{\bf d}}
\newcommand{\bbE}{{\bf E}}
\newcommand{\bbe}{{\bf e}}
\newcommand{\bbI}{{\bf I}}
\newcommand{\bbi}{{\bf i}}
\newcommand{\bbJ}{{\bf J}}
\newcommand{\bbj}{{\bf j}}
\newcommand{\bbK}{{\bf K}}
\newcommand{\bbk}{{\bf k}}
\newcommand{\bbP}{{\bf P}}
\newcommand{\bbp}{{\bf p}}
\newcommand{\bbQ}{{\bf Q}}
\newcommand{\bbq}{{\bf q}}
\newcommand{\bbT}{{\bf T}}
\newcommand{\bbt}{{\bf t}}
\newcommand{\bbU}{{\bf U}}
\newcommand{\bbu}{{\bf u}}
\newcommand{\bbV}{{\bf V}}
\newcommand{\bbv}{{\bf v}}
\newcommand{\bbW}{{\bf W}}
\newcommand{\bbw}{{\bf w}}
\newcommand{\bbX}{{\bf X}}
\newcommand{\bbx}{{\bf x}}
\newcommand{\X}{{\cal X}}
\newcommand{\bbY}{{\bf Y}}
\newcommand{\bby}{{\bf y}}
\newcommand{\bbZ}{{\bf Z}}
\newcommand{\bbz}{{\bf z}}
\newcommand{\0}{{\bf 0}}
\newcommand{\R}{{\bf R}}
\newcommand{\txi}{\bar{\xi}}
\def\Comment#1{ \marginpar{$\bullet$\quad{\tiny #1}}}


\usepackage{graphics,graphicx,amsmath,float,color,subfigure,enumerate,booktabs}
%\usepackage[tiling]{pst-fill}
\usepackage[dvips]{xy}
\usepackage{tikz}
\usetikzlibrary{matrix}
\input{rgb}
\xyoption{all}


\newcommand{\ra}[1]{\renewcommand{\arraystretch}{#1}}


\begin{document}
%controls the margin
\baselineskip=6.0mm








%Ignore some parts of statements
\newcommand{\ignore}[1]{}{}





%Equation numbers contain section number
\renewcommand{\theequation}{\arabic{section}.\arabic{equation}}





%Activate this if I want to display eq number only
\newcommand{\lbl}{\label}

%Activate this if I want to display eq number and text number, too
%\newcommand{\lbl}[1]{\hspace{1cm} \underline{({#1})} \label{#1}}





%Call this eq numbers in text
\newcommand{\eq}[1]{$(\ref{#1})$}


\newcommand{\f}{\frac}    


%Short for Greek letters
\newcommand{\al}{\alpha}                         %\al=\al
\newcommand{\bt}{\beta}                          %\bt=w
\newcommand{\ga}{\gamma}                         %\ga=\gamma
\newcommand{\Ga}{\Gamma}                         %\Ga=\Gamma
\newcommand{\de}{\delta}                         %\de=\delta
\newcommand{\De}{\Delta}                         %\De=\Delta
\newcommand{\ep}{\epsilon}                       %\ep=\epilon
\newcommand{\ve}{\varepsilon}                    %\ve=\varepsilon
\newcommand{\la}{\lambda}                        %\la=\lambda
\newcommand{\La}{\Lambda}                        %\La=\Lambda
\newcommand{\ro}{\rho}                           %\ro=\rho
\newcommand{\ta}{\tau}                           %\ta=tau
%\newcommand{\th}{\theta}                         %\th=\theta
\newcommand{\si}{\sigma}                         %\si=\sigma
\newcommand{\Si}{\Sigma}                         %\si=\sigma
\newcommand{\om}{\omega}                           %\ro=\rho
\newcommand{\Om}{\Omega}                           %\ta=tau





%Short for equation array and for equations
%%%%%%%%%%%%%%%%%%%%%%%%%%%%%%%%%%%%
\newcommand{\be}{\begin{equation}}               %\be=\begin{equation}
\newcommand{\ee}{\end{equation}}                 %\ee=\end{equation}
%%%%%%%%%%%%%%%%%%%%%%%%%%%%%%%%%%%%
\newcommand{\bea}{\begin{eqnarray}}              %\bea=\begin{eqnarray}
\newcommand{\eea}{\end{eqnarray}}                %\eea=\end{eqnarray}
%%%%%%%%%%%%%%%%%%%%%%%%%%%%%%%%%%%%
\newcommand{\bean}{\begin{eqnarray*}}            %\beq=\begin{eqnarray*}
\newcommand{\eean}{\end{eqnarray*}}              %\eeq=\end{eqnarray*}
%%%%%%%%%%%%%%%%%%%%%%%%%%%%%%%%%%%%
%\newcommand{\beq}{\begin{eqnarray*}}            %\beq=\begin{eqnarray*}
%\newcommand{\eeq}{\end{eqnarray*}}              %\eeq=\end{eqnarray*}
%%%%%%%%%%%%%%%%%%%%%%%%%%%%%%%%%%%%
\newcommand{\ba}{\begin{array}}                  %\ba=\begin{array}
\newcommand{\ea}{\end{array}}                    %\ea=\end{array}
%%%%%%%%%%%%%%%%%%%%%%%%%%%%%%%%%%%%
\newcommand{\nn}{\nonumber}                      %\nn=\nonumber
\newcommand{\mb}{\mbox}                          %\mb=\mbox





%\newcommand{\ra}{\rightarrow}                    %\ra=\rightarrow
\newcommand{\Ra}{\Rightarrow}                    %\ra=\rightarrow
\newcommand{\ua}{\uparrow}   
\newcommand{\da}{\downarrow}   
\newcommand{\Lra}{\Leftrightarrow}          %\llra=\longleftrightarrow
\newcommand{\llra}{\longleftrightarrow}          %\llra=\longleftrightarrow

\newcommand{\stac}{\stackrel}                    %\stac=\stackrel
\newcommand{\noin}{\noindent}                    %\noin=\noindent

%Symbol for the end of the proof
\newcommand{\qed}{\nobreak\quad\vrule width6pt depth3pt height10pt}

\newcommand{\ngi}{n \ra \infty}

%heading
%\pagestyle{myheadings} \markright{Rooted edges of a minimal directed spanning tree on random points}
\pagestyle{myheadings} \markright{[Homework] Counting}

\thispagestyle{plain}


\begin{center}
{\Large\bf [Homework] Counting} 
\end{center}





\begin{enumerate}





\item
How many different letter arrangements can be made from 
\begin{enumerate}
\item cat?
\\
{\color{blue}{\bf Sol.}}
Since “cat” has three unique letters we have 3! = 6 possible arrangements.
\item bob?
\\
{\color{blue}{\bf Sol.}}
Since “bob” has three characters with the \textquotedblleft b \textquotedblright repeated two times, so we have $\f{3!}{2!}$ possible arrangements.
\item paper?
\\
{\color{blue}{\bf Sol.}}
Since “paper” has five characters with the \textquotedblleft p \textquotedblright repeated two times, so we have $\f{5!}{2!}$ possible arrangements.
\item pepper?
\\
{\color{blue}{\bf Sol.}}
Since “pepper” has six characters with the \textquotedblleft p \textquotedblright repeated three times, the \textquotedblleft e \textquotedblright
repeated two times so we have $\f{6!}{2!\cdot3!}$ possible arrangements.
\item Mississippi?
\\
{\color{blue}{\bf Sol.}}
Since “Mississippi” has eleven characters with the \textquotedblleft i \textquotedblright repeated four times, the \textquotedblleft s \textquotedblright
repeated four times and the \textquotedblleft p \textquotedblright repeated two times, so we have $\f{11!}{4!\cdot4!\cdot2!}$ possible arrangements.
\end{enumerate}





\item
Delegates from 10 countries, including Korea, Japan, China, and the United States, are to be seated in a row. 
How many different seating arrangements are possible if 
\begin{enumerate}
\item there is no constraint?
\\
{\color{blue}{\bf Sol.}}
$10!$

\item Korea and Japan delegates are to be seated next to each other?
\\
{\color{blue}{\bf Sol.}}
$9!\cdot2!$
\item Korea and Japan delegates are to be seated next to each other and the China and U.S. delegates are not to be next to each other?
\\
{\color{blue}{\bf Sol.}}
If the Korea and Japan delegates are to be seated next to each other, they can be
placed in $2!$ ways. Then this pair constitutes a new object which we can place anywhere
among the remaining eight people, i.e. there are $9!$ arrangements of the eight remaining
people and the Korea and Japan pair. Thus we have $2\cdot9! = 725760$ possible combinations.
Since in some of these the China and U.S. delegates are next to each other, this number
over counts the true number we are looking for by $2 \cdot 2\cdot8!$ = 161280 (the first two is for the
number of arrangements of the China and U.S. pair). Combining these two criterion we
have $2\cdot9! - 2 \cdot 2\cdot8!$.
\end{enumerate}





\item
An elevator starts at the basement of the 12 story building with 10 people
and each one choose one's stop. 
How many different stop arrangements are possible if 
\begin{enumerate}
\item there is no constraint?
\\
{\color{blue}{\bf Sol.}}
Let $x_i$ equal the number of people getting off at floor $i$, where $i = 1,\cdots,12$.
Then the constraint that all people are off at the twelveth floor means that $x_1+ \cdots +x_{12}=10$ with $x_i \ge 0 $. 
\\
This has ${n+r-1\choose r-1}={10+12-1\choose 12-1}={21\choose 11}$ possible distribution people.
\item three are from the same family and they move as a group?
\\
{\color{blue}{\bf Sol.}}
Three people are from the same family means that $x_1+ \cdots +x_{12}=8.$
This has ${n+r-1\choose r-1}={8+12-1\choose 12-1}={19\choose 11}$ possible distribution people.
\item three move as a group and another four move as a group, but these two groups stop at different floors?
\\
{\color{blue}{\bf Sol.}}
This means that \\
(the number of non-negative integer solution $x_1+ \cdots +x_{12}=5$) - (the number of non-negative integer solution $x_1+ \cdots +x_{12}=4$)\\
${5+12-1\choose 12-1}-{4+12-1\choose 12-1}={16\choose 11}-{15\choose 11}$

\end{enumerate}





\item
12 people are to be divided into 3 groups.
How many different group formations are possible if the size of 3 groups are 
\begin{enumerate}
\item 3, 4, and 5?
\\
{\color{blue}{\bf Sol.}}
There are ${12\choose 3}\cdot{9\choose 4}\cdot{5\choose 5}={12\choose 3,4,5}$ possible divisions.
\item 3, 3, and 6? {\color{red}Here, we assume two groups of size 3 are not distinguishable.}
\\
{\color{blue}{\bf Sol.}}
There are $\f{ {12\choose 3}\cdot{9\choose 3}\cdot{6\choose 6}}{2!}$ possible divisions.

\item 3, 3, and 6? {\color{red}Here, we assume (different tasks are assigned to each group and hence) two groups of size 3 are distinguishable.} 
\\
{\color{blue}{\bf Sol.}}
There are ${{12\choose 3}\cdot{9\choose 3}\cdot{6\choose 6}}$ possible divisions.
\end{enumerate}



\item
When we expand $(x+2y+3z+4w)^{100}$,
find the coefficient of
\begin{enumerate}
\item $x^{10}y^{20}z^{30}w^{39}$.
\\
{\color{blue}{\bf Sol.}} 0


\item $x^{10}y^{20}z^{30}w^{40}$.
\\
{\color{blue}{\bf Sol.}} by multinomial theorem\\
 ${100 \choose 10,20,30,40}2^{20}\cdot3^{30}\cdot4^{40}=\f{100!}{10!20!30!40!}2^{20}\cdot3^{30}\cdot4^{40}={100 \choose 10}\cdot{90\choose 20}\cdot{70\choose 30}\cdot{40\choose 40}\cdot2^{20}\cdot3^{30}\cdot4^{40}$

 
 
 
\item $x^{10}y^{20}z^{30}w^{41}$.
\\
{\color{blue}{\bf Sol.}} 0
\end{enumerate}





\item 
If 20 balls are to be distributed into 4 distinct urns, 
how many different configurations are possible if
\begin{enumerate}
\item 20 balls are all different.
\\
{\color{blue}{\bf Sol.}}
$4^{20}$

\item 20 balls are all identical.
\\
{\color{blue}{\bf Sol.}}
If we let $x_i, i = 1, 2, 3, 4$, $i$-th urn, then $x_1, x_2, x_3, x_4$ are integers satisfying the equation $x_1+x_2+x_3+x_4=20, x_i \ge 0$
\\So we have ${20+4-1\choose 4-1}={23\choose 3}$ possible configurations.

\item 20 balls are all identical and each urn has at least one ball.
\\
{\color{blue}{\bf Sol.}}
If we let $x_i, i = 1, 2, 3, 4$, $i$-th urn, then $x_1, x_2, x_3, x_4$ are integers satisfying the equation $x_1+x_2+x_3+x_4=20, x_i > 0$
\\
let $x_i^{'}=x_i-1$
\\
$\Leftrightarrow $  $x_1^{'}+x_2^{'}+x_3^{'}+x_4^{'}=16, x_i \ge 0$
\\So we have ${16+4-1\choose 4-1}={19\choose 3}$ possible configurations.
\item 12 balls are identically red and 8 balls are identically white.
\\
{\color{blue}{\bf Sol.}}
(the number of non-negative integer solutions of $x_1+x_2+x_3+x_4=12)\cdot$
(the number of non-negative integer solutions of $y_1+y_2+y_3+y_4=8$)
\\So we have ${12+4-1\choose 4-1}\cdot {8+4-1\choose 4-1}={15\choose 3}\cdot {11\choose 3}$
\end{enumerate}





































\item %changed from Ross
Prove
$$
{2n\choose n}=\sum_{k=0}^n{n\choose k}^2
$$
\\
{\color{blue}{\bf Sol.}}
From problem 8 we have that when $m = n$ and $r = n$ that
$$
{n+n\choose n}={n\choose 0}{n\choose n}+{n\choose 1}{n\choose n-1}+\cdots+{n\choose n}{n\choose 0}$$
Using the fact that ${n\choose k}={n\choose n-k}$ the above is becomes
$$
{2n\choose n}={n\choose 0}^2+{n\choose 1}^2+\cdots+{n\choose n}^2$$



\item  
Prove that
$$
{n+m\choose r}={n\choose 0}{m\choose r}+{n\choose 1}{m\choose r-1}+\cdots+{n\choose r}{m\choose 0}$$
\\
{\color{blue}{\bf Sol.}}
We can do this in a combinatorial way by considering subgroups of size $r$ from a group of $n$ men and $m$ women. The left hand side of the above represents one way of obtaining this identity. Another way to count the number of subsets of size $r$ is to consider the number of possible groups can be found by considering a subproblem of how many men chosen to be included in the subset of size $r$. This number can range from zero men to $r$ men. When
we have a subset of size $r$ with zero men we must have all women. This can be done in
${n\choose 0}{m\choose r}$ ways. If we select one man and $r-1$ women the number of subsets that meet this criterion is given by ${n\choose 1}{m\choose r-1}$. Continuing this logic for all possible subset of
the men we have the right hand side of the above expression.


\item 
Show the following combinatorial identities
\bea
(a)\sum_{k=1}^n k  {n \choose k}&=&n2^{n-1}\nn\\
(b)\sum_{k=0}^n k(k-1)  {n \choose k}&=&n(n-1)2^{n-2}\nn
\eea
\\
{\color{blue}{\bf Sol.}}
(a) Consider $n$ people from which we want to count the total number of committees
of any size with a chairman. For a committee of size $k = 1$ we have $1\cdot{n\choose 0}=n$ possible choices. For a committee of size $k = 2$ we have ${n\choose 2}$ subsets of two people and two choices
for the person who is the chair. This gives $2\cdot {n\choose 2}$ possible choices. For a committee of size
$k = 3$ we have $3\cdot {n\choose 3}$ etc. Summing all of these possible choices we find that the total
number of committees with a chair is $\sum_{k=1}^n k  {n \choose k}$
\\
Another way to count the total number of all committees with a chair, is to consider first
selecting the chairperson from which we have $n$ choices and then considering all possible
subsets of size $n - 1$ (which is $2^{n-1}$) from which to construct the remaining committee
members. The product then gives $n2^{n-1}$.
\\

(b) Consider again $n$ people where now we want to count the total number of committees of size $k$ with a chairperson and a secretary. We can select all subsets of size $k$ in ${n\choose k}$ ways. Given a subset of size $k$, there are $k$ choices for the chairperson and $k-1$ choices for the secretary giving $k(k-1){n\choose k}$
committees of size $k$ with a chair and a secretary. The
total number of these is then given by summing this result or 
$$\sum_{k=0}^n k(k-1){n \choose k}$$

Now consider first selecting the chairman which can be done in $n$ ways. And selecting for the secretary which can be done in $n-1$ ways
and then we look for all subsets of a set with $n - 2$ elements (i.e. $2^{n-2}$)
\\So we have 
$$\sum_{k=0}^n k(k-1)  {n \choose k} = n(n-1)2^{n-2}$$
\item  
Let $A_n$, $B_n$, and $C_n$ be the number of $n$ coin flip outcomes 
such that, when we divide the number of heads by 3,
the remainder is 0, 1, 2, respectively.  
Show that
$$A_n=A_{n-1}+C_{n-1},\quad \mb{for $n\ge 2$}$$
$$B_n=B_{n-1}+A_{n-1},\quad \mb{for $n\ge 2$}$$
$$C_n=C_{n-1}+B_{n-1},\quad \mb{for $n\ge 2$}$$
or
$$
\left[\ba{c}A_n\\B_n\\C_n\ea\right]
=
\left[\ba{ccc}1&0&1\\1&1&0\\0&1&1\ea\right]
\left[\ba{c}A_{n-1}\\B_{n-1}\\C_{n-1}\ea\right]
$$
Since $A_1=1,\ B_1=1,\ C_1=0$,
$$
\left[\ba{c}A_n\\B_n\\C_n\ea\right]
=
\left[\ba{ccc}1&0&1\\1&1&0\\0&1&1\ea\right]^{n-1}
\left[\ba{c}1\\1\\0\ea\right]
$$
\\
{\color{blue}{\bf Sol.}}
\\
(1) $A_n$ = the number of heads is the form of $3k$  
If $n$th flip is head, the number of heads is the form of $3k+2$ in $n-1$ flips.
If $n$th flip is tail, the number of heads is the form of $3k$ in $n-1$ flips.
So we have $A_n=A_{n-1}+C_{n-1}$
\\
(2) $B_n$ = the number of heads is the form of $3k+1$ 
If $n$th flip is head, the number of heads is the form of $3k$ in $n-1$ flips.
If $n$th flip is tail, the number of heads is the form of $3k+1$ in $n-1$ flips.
So we have $B_n=B_{n-1}+A_{n-1}$
\\
(3) $C_n$ = the number of heads is the form of $3k+2$ 
If $n$th flip is head, the number of heads is the form of $3k+1$ in $n-1$ flips.
If $n$th flip is tail, the number of heads is the form of $3k+2$ in $n-1$ flips.
So we have $C_n=C_{n-1}+B_{n-1}$
\item
An elevator starts at the basement of the 12 story building with 10 people
and each one choose one's stop. 
Count
\begin{enumerate}
\item all the possible red stop button number configurations.
\\
{\color{blue}{\bf Sol.}}
$${12 \choose 1}+{12 \choose 2} \cdots + {12 \choose 12} $$
\item all the possible stopping configurations of all 10 people.
\\
{\color{blue}{\bf Sol.}}
$$12^10$$
\item all the possible stopping configurations of all 10 people {\color{red}if we cannot identify (or don't care) one from another.}
\\
{\color{blue}{\bf Sol.}}
$$ {10+12-1\choose 12-1}={21\choose 11} $$
\end{enumerate}

\end{enumerate}







\end{document}
