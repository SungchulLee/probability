\documentclass[12pt]{article}%report, article
%\documentstyle[12pt,leqno]{article}

\textwidth=15.5cm \textheight=21.6cm \topmargin=-0.5cm
\oddsidemargin=0.05cm

\newcommand{\bbA}{{\bf A}}
\newcommand{\bba}{{\bf a}}
\newcommand{\bbB}{{\bf B}}
\newcommand{\bbb}{{\bf b}}
\newcommand{\bbC}{{\bf C}}
\newcommand{\bbc}{{\bf c}}
\newcommand{\bbD}{{\bf D}}
\newcommand{\bbd}{{\bf d}}
\newcommand{\bbE}{{\bf E}}
\newcommand{\bbe}{{\bf e}}
\newcommand{\bbI}{{\bf I}}
\newcommand{\bbi}{{\bf i}}
\newcommand{\bbJ}{{\bf J}}
\newcommand{\bbj}{{\bf j}}
\newcommand{\bbK}{{\bf K}}
\newcommand{\bbk}{{\bf k}}
\newcommand{\bbP}{{\bf P}}
\newcommand{\bbp}{{\bf p}}
\newcommand{\bbQ}{{\bf Q}}
\newcommand{\bbq}{{\bf q}}
\newcommand{\bbT}{{\bf T}}
\newcommand{\bbt}{{\bf t}}
\newcommand{\bbU}{{\bf U}}
\newcommand{\bbu}{{\bf u}}
\newcommand{\bbV}{{\bf V}}
\newcommand{\bbv}{{\bf v}}
\newcommand{\bbW}{{\bf W}}
\newcommand{\bbw}{{\bf w}}
\newcommand{\bbX}{{\bf X}}
\newcommand{\bbx}{{\bf x}}
\newcommand{\X}{{\cal X}}
\newcommand{\bbY}{{\bf Y}}
\newcommand{\bby}{{\bf y}}
\newcommand{\bbZ}{{\bf Z}}
\newcommand{\bbz}{{\bf z}}
\newcommand{\0}{{\bf 0}}
\newcommand{\R}{{\bf R}}
\newcommand{\txi}{\bar{\xi}}
\def\Comment#1{ \marginpar{$\bullet$\quad{\tiny #1}}}


\usepackage{graphics,graphicx,amsmath,float,color,subfigure,enumerate,booktabs}
%\usepackage[tiling]{pst-fill}
\usepackage[dvips]{xy}
\usepackage{tikz}
\usetikzlibrary{matrix}
\input{rgb}
\xyoption{all}


\newcommand{\ra}[1]{\renewcommand{\arraystretch}{#1}}


\begin{document}
%controls the margin
\baselineskip=6.0mm








%Ignore some parts of statements
\newcommand{\ignore}[1]{}{}





%Equation numbers contain section number
\renewcommand{\theequation}{\arabic{section}.\arabic{equation}}





%Activate this if I want to display eq number only
\newcommand{\lbl}{\label}

%Activate this if I want to display eq number and text number, too
%\newcommand{\lbl}[1]{\hspace{1cm} \underline{({#1})} \label{#1}}





%Call this eq numbers in text
\newcommand{\eq}[1]{$(\ref{#1})$}


\newcommand{\f}{\frac}    


%Short for Greek letters
\newcommand{\al}{\alpha}                         %\al=\al
\newcommand{\bt}{\beta}                          %\bt=w
\newcommand{\ga}{\gamma}                         %\ga=\gamma
\newcommand{\Ga}{\Gamma}                         %\Ga=\Gamma
\newcommand{\de}{\delta}                         %\de=\delta
\newcommand{\De}{\Delta}                         %\De=\Delta
\newcommand{\ep}{\epsilon}                       %\ep=\epilon
\newcommand{\ve}{\varepsilon}                    %\ve=\varepsilon
\newcommand{\la}{\lambda}                        %\la=\lambda
\newcommand{\La}{\Lambda}                        %\La=\Lambda
\newcommand{\ro}{\rho}                           %\ro=\rho
\newcommand{\ta}{\tau}                           %\ta=tau
%\newcommand{\th}{\theta}                         %\th=\theta
\newcommand{\si}{\sigma}                         %\si=\sigma
\newcommand{\Si}{\Sigma}                         %\si=\sigma
\newcommand{\om}{\omega}                           %\ro=\rho
\newcommand{\Om}{\Omega}                           %\ta=tau





%Short for equation array and for equations
%%%%%%%%%%%%%%%%%%%%%%%%%%%%%%%%%%%%
\newcommand{\be}{\begin{equation}}               %\be=\begin{equation}
\newcommand{\ee}{\end{equation}}                 %\ee=\end{equation}
%%%%%%%%%%%%%%%%%%%%%%%%%%%%%%%%%%%%
\newcommand{\bea}{\begin{eqnarray}}              %\bea=\begin{eqnarray}
\newcommand{\eea}{\end{eqnarray}}                %\eea=\end{eqnarray}
%%%%%%%%%%%%%%%%%%%%%%%%%%%%%%%%%%%%
\newcommand{\bean}{\begin{eqnarray*}}            %\beq=\begin{eqnarray*}
\newcommand{\eean}{\end{eqnarray*}}              %\eeq=\end{eqnarray*}
%%%%%%%%%%%%%%%%%%%%%%%%%%%%%%%%%%%%
%\newcommand{\beq}{\begin{eqnarray*}}            %\beq=\begin{eqnarray*}
%\newcommand{\eeq}{\end{eqnarray*}}              %\eeq=\end{eqnarray*}
%%%%%%%%%%%%%%%%%%%%%%%%%%%%%%%%%%%%
\newcommand{\ba}{\begin{array}}                  %\ba=\begin{array}
\newcommand{\ea}{\end{array}}                    %\ea=\end{array}
%%%%%%%%%%%%%%%%%%%%%%%%%%%%%%%%%%%%
\newcommand{\nn}{\nonumber}                      %\nn=\nonumber
\newcommand{\mb}{\mbox}                          %\mb=\mbox





%\newcommand{\ra}{\rightarrow}                    %\ra=\rightarrow
\newcommand{\Ra}{\Rightarrow}                    %\ra=\rightarrow
\newcommand{\ua}{\uparrow}   
\newcommand{\da}{\downarrow}   
\newcommand{\Lra}{\Leftrightarrow}          %\llra=\longleftrightarrow
\newcommand{\llra}{\longleftrightarrow}          %\llra=\longleftrightarrow

\newcommand{\stac}{\stackrel}                    %\stac=\stackrel
\newcommand{\noin}{\noindent}                    %\noin=\noindent

%Symbol for the end of the proof
\newcommand{\qed}{\nobreak\quad\vrule width6pt depth3pt height10pt}

\newcommand{\ngi}{n \ra \infty}

%heading
%\pagestyle{myheadings} \markright{Rooted edges of a minimal directed spanning tree on random points}
\pagestyle{myheadings} \markright{[Homework] Variance}

\thispagestyle{plain}


\begin{center}
{\Large\bf [Homework] Variance} 
\end{center}






\begin{enumerate}




\item
A certain project will be undertaken in 6 stages. There is a 95\% chance that each stage will be completed on time independently.
\begin{enumerate}
\item
Compute the probability that all 6 stages are completed on time.
\\
{\color{blue}{\bf Sol.}}
Let $X_i$ be the $i$th stage.
$$
X_i=\left\{\ba{ll}
1 &\ p=0.95 \\
0 &\ q=0.05\
\ea\right.
$$
$$P(X_1) \cdots P(X_6)=(0.95)^6$$

\item
Compute the expectation and variance of the number of stages that will be completed on time.
\\
{\color{blue}{\bf Sol.}}
$X$ be the umber of stages that will be completed on time.
$$X=X_1+ \cdots + X_6$$
$$E(X)=E(X_1+ \cdots + X_6)=6\cdot0.95=5.7$$
$$Var(X)=6\cdot0.95\cdot0.05=0.285$$

\end{enumerate}




\item
Find the mean and variance of $1+2X+3Y$
where
$E(X)= 1$, $Var(X) = 5$, $E(Y) = 2$, $Var(Y) = 2$, and $\rho=0.5$.
\\
{\color{blue}{\bf Sol.}}
$$E[1 + 2X+3Y]=1+2E[X]+3E[X]=1+2\cdot1+3\cdot2=9$$
Since 
$$\rho=\f{Cov(X,Y)}{\sqrt{Var(X)Var(Y)}}$$
$$Cov(X,Y)=0.5\sqrt{10}$$
\bea
Var(1 + 2X+3Y)
&=&Var(2X)+Var(3Y)+2\cdot Cov(2X,3Y) \nn\\
&=&2^2Var(X)+3^2Var(Y)+2\cdot 2 \cdot 3Cov(X,Y) \nn\\
&=&38+6 \sqrt{10} \nn
\eea



\item 
Suppose that a fair coin is rolled twice, resulting 1 (head) or 0 (tail) each time independently. 
\begin{enumerate}
\item Calculate the variance of the maximum value $X$ to appear in the two rolls.
\\
{\color{blue}{\bf Sol.}}
\begin{table}[H]    \center
\begin{tabular}{|c||c|c||c|} \hline
           $i$        &   $0$      &   $1$   & \\
\hline 
$P(X=i)$    & $\f{1}{4}$  & $\f{3}{4}$& 1\\
\hline

\end{tabular}\vspace{3mm}
%\caption{\small Bermudan Put option}
%\label{Vol10}
\end{table}
$$EX=0\cdot\f{1}{4}+1\cdot\f{3}{4}=\f{3}{4}$$
$$VarX=0^2\cdot\f{1}{4}+1^2\cdot\f{3}{4}-(\f{3}{4})^2=\f{3}{16}$$
\item Calculate the variance of the minimum value $Y$ to appear in the two rolls.
\\
{\color{blue}{\bf Sol.}}
\begin{table}[H]    \center
\begin{tabular}{|c||c|c||c|} \hline
           $i$        &   $0$      &   $1$   & \\
\hline 
$P(Y=i)$    & $\f{3}{4}$  & $\f{1}{4}$& 1\\
\hline

\end{tabular}\vspace{3mm}
%\caption{\small Bermudan Put option}
%\label{Vol10}
\end{table}
$$EY=0\cdot\f{3}{4}+1\cdot\f{1}{4}=\f{1}{4}$$
$$VarY=0^2\cdot\f{3}{4}+1^2\cdot\f{1}{4}-(\f{1}{4})^2=\f{3}{16}$$
\item Calculate $Cov(X,Y)$ and check the sign of $Cov(X,Y)$ with your intuition.
\\
{\color{blue}{\bf Sol.}}
\begin{table}[H]    \center
\begin{tabular}{|c||c|c||c|} \hline
           $i$        &   $0$      &   $1$   & \\
\hline 
$P(XY=i)$    & $\f{3}{4}$  & $\f{1}{4}$& 1\\
\hline

\end{tabular}\vspace{3mm}
%\caption{\small Bermudan Put option}
%\label{Vol10}
\end{table}
$$Cov(X,Y)=E(XY)-EX\cdot EY=\f{1}{16}$$
$$VarX=0^2\cdot\f{1}{4}+1^2\cdot\f{3}{4}-(\f{3}{4})^2=\f{3}{16}$$
\item Calculate the variance of the sum of the two rolls.
\\
{\color{blue}{\bf Sol.}}
\begin{table}[H]    \center
\begin{tabular}{|c||c|c|c||c|} \hline
           $i$        &   $0$      &   $1$   & $2$ & \\
\hline 
$P(X=i)$    & $\f{1}{4}$  & $\f{1}{2}$&  $\f{1}{4}$  &$1$\\
\hline

\end{tabular}\vspace{3mm}
%\caption{\small Bermudan Put option}
%\label{Vol10}
\end{table}
$$EX=0\cdot\f{1}{4}+1\cdot\f{1}{2}+2\cdot\f{1}{4}=1$$
$$VarX=0^2\cdot\f{1}{4}+1^2\cdot\f{1}{2}+2^2\cdot\f{1}{4}-1^2=\f{1}{2}$$
\item Calculate the variance of the first roll number minus the second roll number.
\\
{\color{blue}{\bf Sol.}}
\begin{table}[H]    \center
\begin{tabular}{|c||c|c|c||c|} \hline
           $i$        &   $-1$      &   $0$   & $1$ & \\
\hline 
$P(X=i)$    & $\f{1}{4}$  & $\f{1}{2}$&  $\f{1}{4}$  &$1$\\
\hline

\end{tabular}\vspace{3mm}
%\caption{\small Bermudan Put option}
%\label{Vol10}
\end{table}
$$EX=(-1)\cdot\f{1}{4}+0\cdot\f{1}{2}+1\cdot\f{1}{4}=0$$
$$VarX=(-1)^2\cdot\f{1}{4}+0^2\cdot\f{1}{2}+1^2\cdot\f{1}{4}-0^2=\f{1}{2}$$
\end{enumerate}











\item
Let $X_i$ be independent with same mean 0, and variance 2.
Let $Y_n=X_n+X_{n+1}+X_{n+2}$.
For $j\ge 0$, calculate $Cov(Y_n,Y_{n+j})$.
\\
{\color{blue}{\bf Sol.}}
Since $X_i$ is independent

$$
Cov(X_i,X_j)=\left\{\ba{ll}
0 &i\neq j\ \\
2 &i=j\
\ea\right.
$$
If $j=0$, 
$$Cov(Y_n,Y_{n})=Var(Y_n)=Var(X_n+X_{n+1}+X_{n+2})=Var(X_n)+Var(X_{n+1})+Var(X_{n+2})=6$$
If $j=1$, 
$$Cov(Y_n,Y_{n+1})=Var(X_{n+1})+Var(X_{n+2})=4$$
If $j=2$, 
$$Cov(Y_n,Y_{n+2})=Var(X_{n+2})=2$$
If $j\ge 3$, 
$$Cov(Y_n,Y_{n+3})=0$$










\item
Consider a graph having $n$ vertices labeled
$1,2,\ldots,n$, and suppose that, between each of the ${n\choose 2}$ pairs of distinct vertices, an edge is independently present with probability $p$. The degree of vertex $i$, designated as $D_i$, is the number of edges that have vertex $i$ as one of their vertices.
\begin{enumerate}
\item What is the distribution of $D_i$?
\\
{\color{blue}{\bf Sol.}}
Since each edge occurs independently with probability $p$, the number of edges connected to a given node is a binomial with parameters $n-1$ and $p$ ($n-1$ is the number of other nodes. hence the number of possible edges).

\item Find $\rho(D_i , D_j )$, the correlation between $D_i$ and $D_j$.
\\
{\color{blue}{\bf Sol.}}
Let $X_{ij} = 1$ if there is an edge between vertices $i$ and $j$ and $0$ otherwise. Note that $X_{ij}=X_{ji}$. Then
$$D_i = \sum_{\substack{j = 1 \\ j \neq i}}^n X_{ij}$$
Thus, for $i\neq j$
\bea
Cov(D_i, D_j)
&=&Cov(\sum_{k \neq i} X_{ik}, \sum_{k \neq j} X_{jk}) \nn\\
&=&\sum_{k \neq i} \sum_{l \neq j} Cov(X_{ik}, X_{jl})  \nn\\
&=&Cov(X_{ij}, X_{ji}) = Var(X_{ij}) = p(1-p) \nn
\eea

Where the step on the last line follows from noting that $k=j$ and $l = i$ are the only possible values of $k$ and $l$ for which $X_{ik}$ and $X_{jl}$ are not independent. Thus the correlation is then
\bea
\rho(D_i, D_j)
&=&\frac{Cov(D_i, D_j)}{\sqrt{Var(D_i)}\sqrt{Var(D_j)}} \nn\\
&=&\frac{p(1-p)}{(n-1)p(1-p)}   \nn\\
&=&\f{1}{n-1}  (i\neq j)\nn
\eea
for $i=j$, $\rho(D_i, D_j)=1$
\end{enumerate}
















\item
A box contains 5 red and 5 blue marbles. Two marbles are withdrawn randomly. 
If they are the same color, then you win \$1.10; if they are different, then you  lose \$1.00. Calculate the mean and variance of the amount you win.
\\
{\color{blue}{\bf Sol.}}
$$P(X=1.1)=\f{{5 \choose 2}+{5 \choose 2}}{{10 \choose 2}}=\f{4}{9}$$
$$P(X=-1)=\f{{5 \choose 1}\cdot{5 \choose 1}}{{10 \choose 2}}=\f{5}{9}$$
$$E(X)=1.1\cdot\f{4}{9}+(-1)\cdot\f{5}{9}=-\f{1}{15}$$
$$Var(X)=1.1^2\cdot\f{4}{9}+(-1)^2\cdot\f{5}{9}-(-\f{1}{15})^2$$




\item
A sample of 3 items is selected at random from a box containing 20 items of which 4 are defective. 
Find the expectation and variance of the number of defective items in the sample.
\\
{\color{blue}{\bf Sol.}}
Let $X$ be the number of defective items in the sample.\\
Then $X \sim HG(n,m,N)=HG(3,4,20) $\\
$$EX=n\cdot \f{m}{N}=3\cdot\f{4}{20}=\f{3}{5}$$
$$VarX=n\cdot \f{m}{N}\cdot(1-\f{m}{N})\cdot(1-\f{n-1}{N-1})=3\cdot\f{4}{20}\cdot(1-\f{4}{20})\cdot(1-\f{3-1}{20-1})=\f{204}{475}$$





\item
Five distinct numbers are randomly distributed 
to players numbered 1 through 5. 
Whenever two players compare their numbers, 
the one with the higher one is declared the winner. 
Initially, players 1 and 2 compare their numbers; 
the winner then compares her number with that of player 3, and so on. 
Let $X$ denote the number of times player 1 is a winner. 
Find the expectation and variance of $X$.
\\
{\color{blue}{\bf Sol.}}
In Homework4 \#4,
$$P(X=0)=\f{1}{2},P(X=1)=\f{1}{6},P(X=2)=\f{1}{12} $$
$$P(X=3)=\f{1}{20},P(X=4)=\f{1}{5}$$
$$EX=0\cdot\f{1}{2}+1\cdot\f{1}{6}+2\cdot\f{1}{12} + 3\cdot\f{1}{20}+4\cdot\f{1}{5}=\f{77}{60}$$
$$VarX=0^2\cdot\f{1}{2}+1^2\cdot\f{1}{6}+2^2\cdot\f{1}{12} + 3^2\cdot\f{1}{20}+4^2\cdot\f{1}{5}-(\f{77}{60})^2$$





\item
We mix the ordinary deck of the 52 cards and choose two cards.
If we have 2 aces, we stop.
Otherwise we mix the deck and choose two cards again.
We do this until we get 2 aces.
What is the expectation and variance of the number of trials to get the 2 aces?
\\
{\color{blue}{\bf Sol.}}
$$p=\f{{4 \choose 2}}{{52 \choose 2}}=\f{1}{221}$$
Since $X \sim Geo(p)$
$$EX=\f{1}{p}=221$$
$$VarX=\f{q}{p^2}=221\cdot220=48620$$




\item
Someone I know claims to be able to flip a coin in such a way that he can make it land head 90\% of the time, on the average. I want to test the hypothesis that he's bluffing against the alternative that he is right. I propose to test this hypothesis by having him flip the coin again and again until it first lands tail. If it takes more than 4 tries, I'll conclude that he's right. Assume that the flips are independent.
\begin{enumerate}
\item
Under the null hypothesis $p=0.5$ that he cannot influence the outcome, 
identify the distribution of the number of spins until the coin lands tail.
\\
{\color{blue}{\bf Sol.}}
Let $X$ be the number of spins until the coin lands tail under the null hypothesis.\\
Then $X \sim Geo(0.5)$
\item
What is the expectation and variance of the number of spins to the first tail under the alternative hypothesis $p=0.9$(head)?
\\
{\color{blue}{\bf Sol.}}
Let $Y$ be the number of spinsto the first tail under the alternative hypothesis.\\
Then $Y \sim Geo(0.1)$
$$EY=\f{1}{p}=10$$
$$VarY=\f{q}{p^2}=90$$

\end{enumerate}


\end{enumerate}
\vskip 0.3in





\begin{center}
{\Large\bf Extra} 
\end{center}



\begin{enumerate}
\item
We flips a $p$-coin many times, where $p=0.40$.
\bea
X_i&&\mb{$i^{th}$ flip record,
where $H$ and $T$ are recorded as 1 and 0}\nn\\
Y_i:=2X_i-1&&\mb{$i^{th}$ flip record,
where $H$ and $T$ are recorded as 1 and $-1$}\nn
\eea

Calculate the mean and variance of the following related random variables,
i.e., fill up blanks of the below table.
\begin{table}[H]    \center
\begin{tabular}{|c||c|c|c||c|} \hline
Random variable&Mean&Variance\\\hline\hline  
$Y_i$&&\\\hline
$\sum_{i=1}^nY_i$&&\\\hline
$\f{\sum_{i=1}^nY_i}{\sqrt{n}}$&&\\\hline
\end{tabular}
\end{table}




\item
On a multiple-choice exam with 4 possible answers for each of the 20 questions, 
what is the mean and variance of the number of correct answers 
that a student will get just by guessing?



\item
20 people consisting of 10 couples are in an island.
Each person lives after 1 year with probability 0.5, independently.
Let $X$ be the number of surviving couples after 1 year. 
\begin{enumerate}
\item
Identify the distribution of $X$. 
\\
{\color{blue}{\bf Sol.}}
$$
X_i=\left\{\ba{ll}
1 &i\mb{th couple survive}\ p=\f{1}{2}\cdot\f{1}{2} \\
0 &\mb{otherwise}\ q=1-\f{1}{4}\
\ea\right.
$$
Then $X_i \sim B(\f{1}{4})$
$$X=X_1+X_2+\cdots+X_{10}$$
$$X\sim B(10,\f{1}{4})$$
\item
Calculate the mean and variance of $X$. 
\\
{\color{blue}{\bf Sol.}}
$$E(X)=np=10\cdot\f{1}{4}=\f{5}{2}$$
$$Var(X)=npq=10\cdot\f{1}{4}\cdot\f{3}{4}=\f{15}{8}$$
\end{enumerate}



\item
An urn contains 4 white and 4 black balls. 
We randomly choose 4 balls. 
If 2 of them are white and 2 are black, we stop. 
If not, we replace the balls in the urn and again randomly select 4 balls. 
This continues until exactly 2 of the 4 chosen are white. 
What is the mean and variance of the number of trials?





\item
Let $X_i$, $1\le i \le 4$, have mean 3 and variance 2
and the correlation $\rho_{ij}$ between two are all 0.5.
Compute 
\begin{enumerate}
\item $Cov(X_1 +X_2,X_2 +X_3)$. 
\\
{\color{blue}{\bf Sol.}}
\bea
Cov(X_1 +X_2,X_2 +X_3)
&=&Cov(X_1,X_2)+Cov(X_1,X_3)+Cov(X_2,X_2)+Cov(X_2,X_3) \nn\\
&=&\rho \{\sqrt{Var(X_1)Var(X_2)}+\sqrt{Var(X_1)Var(X_3)} \nn\\
&&\ +Var(X_2)+\sqrt{Var(X_2)Var(X_3)} \} \nn\\
&=&0.5(2+2+2+2)=4 \nn
\eea
\item $Cov(X_1+X_2,X_3+X_4)$.
\\
{\color{blue}{\bf Sol.}}
\bea
Cov(X_1 +X_2,X_3 +X_4)
&=&Cov(X_1,X_3)+Cov(X_1,X_4)+Cov(X_2,X_3)+Cov(X_2,X_4) \nn\\
&=&\rho \{\sqrt{Var(X_1)Var(X_3)} +\sqrt{Var(X_1)Var(X_4)} \nn\\
&&\ +\sqrt{Var(X_2)Var(X_3)}+\sqrt{Var(X_2)Var(X_4)} \} \nn\\
&=&0.5(2+2+2+2)=4 \nn
\eea
\end{enumerate}


\end{enumerate}
\vskip 0.3in

\end{document}
