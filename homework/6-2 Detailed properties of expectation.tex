\documentclass[12pt]{article}%report, article
%\documentstyle[12pt,leqno]{article}

\textwidth=15.5cm \textheight=21.6cm \topmargin=-0.5cm
\oddsidemargin=0.05cm

\newcommand{\bbA}{{\bf A}}
\newcommand{\bba}{{\bf a}}
\newcommand{\bbB}{{\bf B}}
\newcommand{\bbb}{{\bf b}}
\newcommand{\bbC}{{\bf C}}
\newcommand{\bbc}{{\bf c}}
\newcommand{\bbD}{{\bf D}}
\newcommand{\bbd}{{\bf d}}
\newcommand{\bbE}{{\bf E}}
\newcommand{\bbe}{{\bf e}}
\newcommand{\bbI}{{\bf I}}
\newcommand{\bbi}{{\bf i}}
\newcommand{\bbJ}{{\bf J}}
\newcommand{\bbj}{{\bf j}}
\newcommand{\bbK}{{\bf K}}
\newcommand{\bbk}{{\bf k}}
\newcommand{\bbP}{{\bf P}}
\newcommand{\bbp}{{\bf p}}
\newcommand{\bbQ}{{\bf Q}}
\newcommand{\bbq}{{\bf q}}
\newcommand{\bbT}{{\bf T}}
\newcommand{\bbt}{{\bf t}}
\newcommand{\bbU}{{\bf U}}
\newcommand{\bbu}{{\bf u}}
\newcommand{\bbV}{{\bf V}}
\newcommand{\bbv}{{\bf v}}
\newcommand{\bbW}{{\bf W}}
\newcommand{\bbw}{{\bf w}}
\newcommand{\bbX}{{\bf X}}
\newcommand{\bbx}{{\bf x}}
\newcommand{\X}{{\cal X}}
\newcommand{\bbY}{{\bf Y}}
\newcommand{\bby}{{\bf y}}
\newcommand{\bbZ}{{\bf Z}}
\newcommand{\bbz}{{\bf z}}
\newcommand{\0}{{\bf 0}}
\newcommand{\R}{{\bf R}}
\newcommand{\txi}{\bar{\xi}}
\def\Comment#1{ \marginpar{$\bullet$\quad{\tiny #1}}}


\usepackage{graphics,graphicx,amsmath,float,color,subfigure,enumerate,booktabs}
%\usepackage[tiling]{pst-fill}
\usepackage[dvips]{xy}
\usepackage{tikz}
\usetikzlibrary{matrix}
\input{rgb}
\xyoption{all}


\newcommand{\ra}[1]{\renewcommand{\arraystretch}{#1}}


\begin{document}
%controls the margin
\baselineskip=6.0mm








%Ignore some parts of statements
\newcommand{\ignore}[1]{}{}





%Equation numbers contain section number
\renewcommand{\theequation}{\arabic{section}.\arabic{equation}}





%Activate this if I want to display eq number only
\newcommand{\lbl}{\label}

%Activate this if I want to display eq number and text number, too
%\newcommand{\lbl}[1]{\hspace{1cm} \underline{({#1})} \label{#1}}





%Call this eq numbers in text
\newcommand{\eq}[1]{$(\ref{#1})$}


\newcommand{\f}{\frac}    


%Short for Greek letters
\newcommand{\al}{\alpha}                         %\al=\al
\newcommand{\bt}{\beta}                          %\bt=w
\newcommand{\ga}{\gamma}                         %\ga=\gamma
\newcommand{\Ga}{\Gamma}                         %\Ga=\Gamma
\newcommand{\de}{\delta}                         %\de=\delta
\newcommand{\De}{\Delta}                         %\De=\Delta
\newcommand{\ep}{\epsilon}                       %\ep=\epilon
\newcommand{\ve}{\varepsilon}                    %\ve=\varepsilon
\newcommand{\la}{\lambda}                        %\la=\lambda
\newcommand{\La}{\Lambda}                        %\La=\Lambda
\newcommand{\ro}{\rho}                           %\ro=\rho
\newcommand{\ta}{\tau}                           %\ta=tau
%\newcommand{\th}{\theta}                         %\th=\theta
\newcommand{\si}{\sigma}                         %\si=\sigma
\newcommand{\Si}{\Sigma}                         %\si=\sigma
\newcommand{\om}{\omega}                           %\ro=\rho
\newcommand{\Om}{\Omega}                           %\ta=tau





%Short for equation array and for equations
%%%%%%%%%%%%%%%%%%%%%%%%%%%%%%%%%%%%
\newcommand{\be}{\begin{equation}}               %\be=\begin{equation}
\newcommand{\ee}{\end{equation}}                 %\ee=\end{equation}
%%%%%%%%%%%%%%%%%%%%%%%%%%%%%%%%%%%%
\newcommand{\bea}{\begin{eqnarray}}              %\bea=\begin{eqnarray}
\newcommand{\eea}{\end{eqnarray}}                %\eea=\end{eqnarray}
%%%%%%%%%%%%%%%%%%%%%%%%%%%%%%%%%%%%
\newcommand{\bean}{\begin{eqnarray*}}            %\beq=\begin{eqnarray*}
\newcommand{\eean}{\end{eqnarray*}}              %\eeq=\end{eqnarray*}
%%%%%%%%%%%%%%%%%%%%%%%%%%%%%%%%%%%%
%\newcommand{\beq}{\begin{eqnarray*}}            %\beq=\begin{eqnarray*}
%\newcommand{\eeq}{\end{eqnarray*}}              %\eeq=\end{eqnarray*}
%%%%%%%%%%%%%%%%%%%%%%%%%%%%%%%%%%%%
\newcommand{\ba}{\begin{array}}                  %\ba=\begin{array}
\newcommand{\ea}{\end{array}}                    %\ea=\end{array}
%%%%%%%%%%%%%%%%%%%%%%%%%%%%%%%%%%%%
\newcommand{\nn}{\nonumber}                      %\nn=\nonumber
\newcommand{\mb}{\mbox}                          %\mb=\mbox





%\newcommand{\ra}{\rightarrow}                    %\ra=\rightarrow
\newcommand{\Ra}{\Rightarrow}                    %\ra=\rightarrow
\newcommand{\ua}{\uparrow}   
\newcommand{\da}{\downarrow}   
\newcommand{\Lra}{\Leftrightarrow}          %\llra=\longleftrightarrow
\newcommand{\llra}{\longleftrightarrow}          %\llra=\longleftrightarrow

\newcommand{\stac}{\stackrel}                    %\stac=\stackrel
\newcommand{\noin}{\noindent}                    %\noin=\noindent

%Symbol for the end of the proof
\newcommand{\qed}{\nobreak\quad\vrule width6pt depth3pt height10pt}

\newcommand{\ngi}{n \ra \infty}

%heading
%\pagestyle{myheadings} \markright{Rooted edges of a minimal directed spanning tree on random points}
\pagestyle{myheadings} \markright{[Homework] Detailed properties of expectation}

\thispagestyle{plain}


\begin{center}
{\Large\bf [Homework] Detailed properties of expectation} 
\end{center}






\begin{enumerate}



\item
Let $X$ represent the number of heads minus the number of tails obtained when a fair coin is tossed $n$ times. 
What is $EX$?

\item
The county hospital is located at the center of a square whose sides are 3 miles wide. If an accident occurs within this square, then the1 hospital sends out an ambulance. The road network is rectangular, so the travel distance from the hospital, whose coordinates are $(0, 0)$, to the point $(x, y)$ is $|x| + |y|$. If an accident occurs at a point that is uniformly distributed in the square, find the expected travel distance of the ambulance.
\\
{\color{blue}{\bf Sol.}}
First, let $Z=|X|+|Y|$ for simplicity. We want to find $E(Z)$. \\
Note that $-\f{3}{2}<X<\f{3}{2}$ and $-\f{3}{2}<X<\f{3}{2}$ since the sides of the square have length $3$. Thus, $|X|<\f{3}{2}$ and $|Y|<\f{3}{2}$.
\bea
E(Z)
&=&E(|X|+|Y|) \nn\\
&=&E|X|+E|Y| \nn\\
&=&\int_{-\f{3}{2}}^{\f{3}{2}} \f{x}{\f{3}{2}-(-\f{3}{2})} dx + \int_{-\f{3}{2}}^{\f{3}{2}} \f{y}{\f{3}{2}-(-\f{3}{2})} dy \nn \\
&=&2\int_{0}^{\f{3}{2}} \f{x}{3} dx + 2\int_{0}^{\f{3}{2}} \f{y}{3} dy \nn \\
&=&\f{3}{2} \nn
\eea

\item
We mix the ordinary deck of the cards and choose two cards.
If we have 2 aces, we stop.
Otherwise we mix the deck and choose two cards again.
We do this until we get 2 aces.
What is the expected number of trials to get 2 aces?
\\
{\color{blue}{\bf Sol.}}



\item
One of the numbers 1 through 10 is randomly chosen. You are to try to guess the number chosen by asking questions with ``yes-no'' answers. Compute the expected number of questions you will need to ask in each of the following two cases:
\begin{enumerate}
\item Your $i$-th question is to be ``Is it $i$?'' $i = 1, 2, 3, 4, 5, 6, 7, 8, 9, 10$.

\item With each question you try to eliminate one-half of the remaining numbers, as nearly as possible.

\end{enumerate}



\item
Two baseball teams $A$ and $B$ with equal skills plays korean series,
where the team who wins 4 games first becomes the champion. 
What is the expected number of games to play?

\item
A player throws a fair die and simultaneously flips a fair coin. 
If the coin lands heads, then she wins \$2, and if tails, then one-half of the value that appears on the die. Determine her expected winnings.
\\
{\color{blue}{\bf Sol.}}
$$\f{1}{2}\cdot \$2 +\f{1}{12}(\$0.5+\$ 1+\cdots+\$3)=\$ \f{15}{8}$$


\item
Gambles are independent, and each one results in
the player being equally likely to win or lose 1 unit. Let W denote the net winnings of a gambler whose strategy is to stop gambling immediately after his first win. Find
\begin{enumerate}
\item $P(W>0)$. 

\item $P(W<0)$.

\item $EW$.
\end{enumerate}

\item
A box contains tickets labeled with the numbers $\{-3, -1, 0, 1, 3\}$. 
In 100 random draws with replacement from the box, 
calculate the expected value of the sum of the positive numbers on the tickets drawn.
 [From SticiGui]


\item
There are $n$ components lined up in a linear arrangement. 
Suppose that each component independently functions with probability $p$. 
Let $X$ be the number of 2 neighboring pairs that both are not functional. 
Calculate $EX$. 




\item % ???
A box contains 5 red and 5 blue marbles. 
Five marbles are withdrawn randomly with replacement. 
Each time you have red you win \$1.10 and
each time you have blue you win $-$\$1.00,
that is, you lose \$1.00. 
Let $X$ be the total amount you win after five trials.
\begin{enumerate}
\item
Let $Y$ be the number of red marbles drawn during 5 trials.
Represent $X$ as a function of $Y$.  

\item
Calculate  $EX$. 
\end{enumerate}


\end{enumerate}



\end{document}