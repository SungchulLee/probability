\documentclass[12pt]{article}%report, article
%\documentstyle[12pt,leqno]{article}

\textwidth=15.5cm \textheight=21.6cm \topmargin=-0.5cm
\oddsidemargin=0.05cm

\newcommand{\bbA}{{\bf A}}
\newcommand{\bba}{{\bf a}}
\newcommand{\bbB}{{\bf B}}
\newcommand{\bbb}{{\bf b}}
\newcommand{\bbC}{{\bf C}}
\newcommand{\bbc}{{\bf c}}
\newcommand{\bbD}{{\bf D}}
\newcommand{\bbd}{{\bf d}}
\newcommand{\bbE}{{\bf E}}
\newcommand{\bbe}{{\bf e}}
\newcommand{\bbI}{{\bf I}}
\newcommand{\bbi}{{\bf i}}
\newcommand{\bbJ}{{\bf J}}
\newcommand{\bbj}{{\bf j}}
\newcommand{\bbK}{{\bf K}}
\newcommand{\bbk}{{\bf k}}
\newcommand{\bbP}{{\bf P}}
\newcommand{\bbp}{{\bf p}}
\newcommand{\bbQ}{{\bf Q}}
\newcommand{\bbq}{{\bf q}}
\newcommand{\bbT}{{\bf T}}
\newcommand{\bbt}{{\bf t}}
\newcommand{\bbU}{{\bf U}}
\newcommand{\bbu}{{\bf u}}
\newcommand{\bbV}{{\bf V}}
\newcommand{\bbv}{{\bf v}}
\newcommand{\bbW}{{\bf W}}
\newcommand{\bbw}{{\bf w}}
\newcommand{\bbX}{{\bf X}}
\newcommand{\bbx}{{\bf x}}
\newcommand{\X}{{\cal X}}
\newcommand{\bbY}{{\bf Y}}
\newcommand{\bby}{{\bf y}}
\newcommand{\bbZ}{{\bf Z}}
\newcommand{\bbz}{{\bf z}}
\newcommand{\0}{{\bf 0}}
\newcommand{\R}{{\bf R}}
\newcommand{\txi}{\bar{\xi}}
\def\Comment#1{ \marginpar{$\bullet$\quad{\tiny #1}}}


\usepackage{graphics,graphicx,amsmath,float,color,subfigure,enumerate,booktabs}
%\usepackage[tiling]{pst-fill}
\usepackage[dvips]{xy}
\usepackage{tikz}
\usetikzlibrary{matrix}
\input{rgb}
\xyoption{all}


\newcommand{\ra}[1]{\renewcommand{\arraystretch}{#1}}


\begin{document}
%controls the margin
\baselineskip=6.0mm








%Ignore some parts of statements
\newcommand{\ignore}[1]{}{}





%Equation numbers contain section number
\renewcommand{\theequation}{\arabic{section}.\arabic{equation}}





%Activate this if I want to display eq number only
\newcommand{\lbl}{\label}

%Activate this if I want to display eq number and text number, too
%\newcommand{\lbl}[1]{\hspace{1cm} \underline{({#1})} \label{#1}}





%Call this eq numbers in text
\newcommand{\eq}[1]{$(\ref{#1})$}


\newcommand{\f}{\frac}    


%Short for Greek letters
\newcommand{\al}{\alpha}                         %\al=\al
\newcommand{\bt}{\beta}                          %\bt=w
\newcommand{\ga}{\gamma}                         %\ga=\gamma
\newcommand{\Ga}{\Gamma}                         %\Ga=\Gamma
\newcommand{\de}{\delta}                         %\de=\delta
\newcommand{\De}{\Delta}                         %\De=\Delta
\newcommand{\ep}{\epsilon}                       %\ep=\epilon
\newcommand{\ve}{\varepsilon}                    %\ve=\varepsilon
\newcommand{\la}{\lambda}                        %\la=\lambda
\newcommand{\La}{\Lambda}                        %\La=\Lambda
\newcommand{\ro}{\rho}                           %\ro=\rho
\newcommand{\ta}{\tau}                           %\ta=tau
%\newcommand{\th}{\theta}                         %\th=\theta
\newcommand{\si}{\sigma}                         %\si=\sigma
\newcommand{\Si}{\Sigma}                         %\si=\sigma
\newcommand{\om}{\omega}                           %\ro=\rho
\newcommand{\Om}{\Omega}                           %\ta=tau





%Short for equation array and for equations
%%%%%%%%%%%%%%%%%%%%%%%%%%%%%%%%%%%%
\newcommand{\be}{\begin{equation}}               %\be=\begin{equation}
\newcommand{\ee}{\end{equation}}                 %\ee=\end{equation}
%%%%%%%%%%%%%%%%%%%%%%%%%%%%%%%%%%%%
\newcommand{\bea}{\begin{eqnarray}}              %\bea=\begin{eqnarray}
\newcommand{\eea}{\end{eqnarray}}                %\eea=\end{eqnarray}
%%%%%%%%%%%%%%%%%%%%%%%%%%%%%%%%%%%%
\newcommand{\bean}{\begin{eqnarray*}}            %\beq=\begin{eqnarray*}
\newcommand{\eean}{\end{eqnarray*}}              %\eeq=\end{eqnarray*}
%%%%%%%%%%%%%%%%%%%%%%%%%%%%%%%%%%%%
%\newcommand{\beq}{\begin{eqnarray*}}            %\beq=\begin{eqnarray*}
%\newcommand{\eeq}{\end{eqnarray*}}              %\eeq=\end{eqnarray*}
%%%%%%%%%%%%%%%%%%%%%%%%%%%%%%%%%%%%
\newcommand{\ba}{\begin{array}}                  %\ba=\begin{array}
\newcommand{\ea}{\end{array}}                    %\ea=\end{array}
%%%%%%%%%%%%%%%%%%%%%%%%%%%%%%%%%%%%
\newcommand{\nn}{\nonumber}                      %\nn=\nonumber
\newcommand{\mb}{\mbox}                          %\mb=\mbox





%\newcommand{\ra}{\rightarrow}                    %\ra=\rightarrow
\newcommand{\Ra}{\Rightarrow}                    %\ra=\rightarrow
\newcommand{\ua}{\uparrow}   
\newcommand{\da}{\downarrow}   
\newcommand{\Lra}{\Leftrightarrow}          %\llra=\longleftrightarrow
\newcommand{\llra}{\longleftrightarrow}          %\llra=\longleftrightarrow

\newcommand{\stac}{\stackrel}                    %\stac=\stackrel
\newcommand{\noin}{\noindent}                    %\noin=\noindent

%Symbol for the end of the proof
\newcommand{\qed}{\nobreak\quad\vrule width6pt depth3pt height10pt}

\newcommand{\ngi}{n \ra \infty}

%heading
%\pagestyle{myheadings} \markright{Rooted edges of a minimal directed spanning tree on random points}
\pagestyle{myheadings} \markright{[Homework] Variance, covariance, and correlation coefficient}

\thispagestyle{plain}


\begin{center}
{\Large\bf [Homework] Variance, covariance, and correlation coefficient} 
\end{center}






\begin{enumerate}




\item
For $X$ and $Y$ with $E[X] = 1$, $Var(X) = 5$, $E[Y] = 2$, $Var(Y) = 2$, and $\rho=0.5$,  find 
\begin{enumerate}
\item $E[1 + 2X+3Y]$

\item $Var(1 + 2X+3Y)$.
\\
{\color{blue}{\bf Sol.}}


\end{enumerate}





 \item
We flips a fair coin many times
and let $X_i$ be the $i$-th flip record,
where $H$ and $T$ are recorded as 1 and 0.
Let $Y_i$ be $Y_i=2X_i-1$, i.e.,
the $i$-th flip record
where $H$ and $T$ are recorded as 1 and $-1$.
Calculate the mean and variance of related random variables,
i.e., fill up the blank in below table.
\begin{table}[H]    \center
\begin{tabular}{|c||c|c|c||c|} \hline
Random variable&Mean&Variance\\\hline\hline  
$Y_i$&$0$&$1$\\\hline
$\sum_{i=1}^nY_i$&$0$&$n$\\\hline
$\f{\sum_{i=1}^nY_i}{\sqrt{n}}$&$0$&$\sqrt{n}$\\\hline
\end{tabular}
\end{table}




\item
Let $X_i$ be independent with same mean 0, variance 2, and
let $Y_n=X_n+X_{n+1}+X_{n+2}$.
For $j\ge 0$, calculate $Cov(Y_n,Y_{n+j})$.


\item
Let $X$ be the number of heads minus the number of tails obtained 
when a fair coin is tossed $n$ times. 
Calculate the mean and variance of $X$. 
\\
{\color{blue}{\bf Sol.}}
Let $i$ be the number of tails. Then $n-i$ is the number of heads and $X=n-2i$.
$$P(X=i)={n \choose i}(\f{1}{2})^i(\f{1}{2})^{n-i}={n \choose i}(\f{1}{2})^n$$
\bea
E(X)
&=&\sum_{x} x\cdot P(X=x) \nn\\
&=&\sum_{i=0}^n (n-2i){n \choose i}(\f{1}{2})^n\nn\\
&=&\{ n\sum_{i=0}^n {n \choose i}-2\sum_{i=0}^n i\cdot{n \choose i}\}\cdot (\f{1}{2})^n\nn\\
&=&\{ n \cdot 2^n-2\cdot n\cdot2^{n-1}\}\cdot (\f{1}{2})^n\nn\\
&=&0\cdot (\f{1}{2})^n=0\nn
\eea



\bea
Var(X)
&=&\sum_{x} x^2\cdot P(X=x)-(EX)^2 \nn\\
&=&\sum_{i=0}^n (n-2i)^2{n \choose i}(\f{1}{2})^n-0^2\nn\\
&=&\{ n^2\sum_{i=0}^n {n \choose i}-4n\sum_{i=0}^n i\cdot{n \choose i}+4\sum_{i=0}^n i^2\cdot{n \choose i}\}\cdot (\f{1}{2})^n\nn\\
&=&\{ n^2 \cdot 2^n-4n\cdot n\cdot2^{n-1}+4\cdot(n^2+n)2^{n-2}\}\cdot (\f{1}{2})^n\nn\\
&=&n\nn
\eea





\item
We flips a fair coin until heads appears 10 times. 
Let $X$ be the number of tails during these flips.
Calculate the mean and variance of $X$ . 
\\
{\color{blue}{\bf Sol.}}
$$Y \sim NB(r=10, p=\f{1}{2})$$
$$X \sim HG $$
\bea
P(X=n)
&=&P(Y=n+10) \nn\\
&=&{ n+10-1 \choose r-1}p^r(1-p)^{(n+10)-r} \nn\\
&=&{ n+9 \choose 9}(\f{1}{2})^{(n+10)}\nn
\eea




\item
If $X_i$, $1\le i \le 4$, have mean 3 and variance 2
and if the correlation $\rho$ between two are all 0.5,
compute 
\begin{enumerate}
\item $Cov(X_1 +X_2,X_2 +X_3)$. 


\item $Cov(X_1+X_2,X_3+X_4)$.


\end{enumerate}




\item
Consider a graph having $n$ vertices labeled
$1,2,\ldots,n$, and suppose that, between each of the ${n\choose 2}$ pairs of distinct vertices, an edge is independently present with probability $p$. The degree of vertex $i$, designated as $D_i$, is the number of edges that have vertex $i$ as one of their vertices.
\begin{enumerate}
\item What is the distribution of $D_i$?

\item Find $\rho(D_i , D_j )$, the correlation between $D_i$ and $D_j$.



\end{enumerate}





\item
A group of eighteen students contains seven business majors; the rest are liberal arts majors. 
Five students will be drawn from the group and let $X$ be the number of business majors in the sample.
\begin{enumerate}
\item Calculate $P(X=3)$.


\item What is the distribution of $X$?


\item What is $EX$?


\item What is $Var(X)$?


\end{enumerate}



\item
A pond contains 100 fish, of which 30 are carp. If 20 fish are caught, what are the mean and variance of the number of carp among the 20? 






\item
A box contains 5 red and 5 blue marbles. Two marbles are withdrawn randomly. 
If they are the same color, then you win \$1.10; if they are different colors, then you win $-$\$1.00. (That is, you lose \$1.00.) Calculate
\begin{enumerate}
\item the expected value of the amount you win.

\item the variance of the amount you win.


\end{enumerate}


\end{enumerate}
\vskip 0.3in




\begin{center}
{\Large\bf [Extra] Variance, covariance, and correlation coefficient} 
\end{center}






\begin{enumerate}


\item
A sample of 3 items is selected at random from a box containing 20 items of which 4 are defective. 
Find the expectation and variance of the number of defective items in the sample.


\end{enumerate}




\end{document}
