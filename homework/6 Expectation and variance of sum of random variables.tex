\documentclass[12pt]{article}%report, article
%\documentstyle[12pt,leqno]{article}

\textwidth=15.5cm \textheight=21.6cm \topmargin=-0.5cm
\oddsidemargin=0.05cm

\newcommand{\bbA}{{\bf A}}
\newcommand{\bba}{{\bf a}}
\newcommand{\bbB}{{\bf B}}
\newcommand{\bbb}{{\bf b}}
\newcommand{\bbC}{{\bf C}}
\newcommand{\bbc}{{\bf c}}
\newcommand{\bbD}{{\bf D}}
\newcommand{\bbd}{{\bf d}}
\newcommand{\bbE}{{\bf E}}
\newcommand{\bbe}{{\bf e}}
\newcommand{\bbI}{{\bf I}}
\newcommand{\bbi}{{\bf i}}
\newcommand{\bbJ}{{\bf J}}
\newcommand{\bbj}{{\bf j}}
\newcommand{\bbK}{{\bf K}}
\newcommand{\bbk}{{\bf k}}
\newcommand{\bbP}{{\bf P}}
\newcommand{\bbp}{{\bf p}}
\newcommand{\bbQ}{{\bf Q}}
\newcommand{\bbq}{{\bf q}}
\newcommand{\bbT}{{\bf T}}
\newcommand{\bbt}{{\bf t}}
\newcommand{\bbU}{{\bf U}}
\newcommand{\bbu}{{\bf u}}
\newcommand{\bbV}{{\bf V}}
\newcommand{\bbv}{{\bf v}}
\newcommand{\bbW}{{\bf W}}
\newcommand{\bbw}{{\bf w}}
\newcommand{\bbX}{{\bf X}}
\newcommand{\bbx}{{\bf x}}
\newcommand{\X}{{\cal X}}
\newcommand{\bbY}{{\bf Y}}
\newcommand{\bby}{{\bf y}}
\newcommand{\bbZ}{{\bf Z}}
\newcommand{\bbz}{{\bf z}}
\newcommand{\0}{{\bf 0}}
\newcommand{\R}{{\bf R}}
\newcommand{\txi}{\bar{\xi}}
\def\Comment#1{ \marginpar{$\bullet$\quad{\tiny #1}}}


\usepackage{graphics,graphicx,amsmath,float,color,subfigure,enumerate,booktabs}
%\usepackage[tiling]{pst-fill}
\usepackage[dvips]{xy}
\usepackage{tikz}
\usetikzlibrary{matrix}
\input{rgb}
\xyoption{all}


\newcommand{\ra}[1]{\renewcommand{\arraystretch}{#1}}


\begin{document}
%controls the margin
\baselineskip=6.0mm








%Ignore some parts of statements
\newcommand{\ignore}[1]{}{}





%Equation numbers contain section number
\renewcommand{\theequation}{\arabic{section}.\arabic{equation}}





%Activate this if I want to display eq number only
\newcommand{\lbl}{\label}

%Activate this if I want to display eq number and text number, too
%\newcommand{\lbl}[1]{\hspace{1cm} \underline{({#1})} \label{#1}}





%Call this eq numbers in text
\newcommand{\eq}[1]{$(\ref{#1})$}


\newcommand{\f}{\frac}    


%Short for Greek letters
\newcommand{\al}{\alpha}                         %\al=\al
\newcommand{\bt}{\beta}                          %\bt=w
\newcommand{\ga}{\gamma}                         %\ga=\gamma
\newcommand{\Ga}{\Gamma}                         %\Ga=\Gamma
\newcommand{\de}{\delta}                         %\de=\delta
\newcommand{\De}{\Delta}                         %\De=\Delta
\newcommand{\ep}{\epsilon}                       %\ep=\epilon
\newcommand{\ve}{\varepsilon}                    %\ve=\varepsilon
\newcommand{\la}{\lambda}                        %\la=\lambda
\newcommand{\La}{\Lambda}                        %\La=\Lambda
\newcommand{\ro}{\rho}                           %\ro=\rho
\newcommand{\ta}{\tau}                           %\ta=tau
%\newcommand{\th}{\theta}                         %\th=\theta
\newcommand{\si}{\sigma}                         %\si=\sigma
\newcommand{\Si}{\Sigma}                         %\si=\sigma
\newcommand{\om}{\omega}                           %\ro=\rho
\newcommand{\Om}{\Omega}                           %\ta=tau





%Short for equation array and for equations
%%%%%%%%%%%%%%%%%%%%%%%%%%%%%%%%%%%%
\newcommand{\be}{\begin{equation}}               %\be=\begin{equation}
\newcommand{\ee}{\end{equation}}                 %\ee=\end{equation}
%%%%%%%%%%%%%%%%%%%%%%%%%%%%%%%%%%%%
\newcommand{\bea}{\begin{eqnarray}}              %\bea=\begin{eqnarray}
\newcommand{\eea}{\end{eqnarray}}                %\eea=\end{eqnarray}
%%%%%%%%%%%%%%%%%%%%%%%%%%%%%%%%%%%%
\newcommand{\bean}{\begin{eqnarray*}}            %\beq=\begin{eqnarray*}
\newcommand{\eean}{\end{eqnarray*}}              %\eeq=\end{eqnarray*}
%%%%%%%%%%%%%%%%%%%%%%%%%%%%%%%%%%%%
%\newcommand{\beq}{\begin{eqnarray*}}            %\beq=\begin{eqnarray*}
%\newcommand{\eeq}{\end{eqnarray*}}              %\eeq=\end{eqnarray*}
%%%%%%%%%%%%%%%%%%%%%%%%%%%%%%%%%%%%
\newcommand{\ba}{\begin{array}}                  %\ba=\begin{array}
\newcommand{\ea}{\end{array}}                    %\ea=\end{array}
%%%%%%%%%%%%%%%%%%%%%%%%%%%%%%%%%%%%
\newcommand{\nn}{\nonumber}                      %\nn=\nonumber
\newcommand{\mb}{\mbox}                          %\mb=\mbox





%\newcommand{\ra}{\rightarrow}                    %\ra=\rightarrow
\newcommand{\Ra}{\Rightarrow}                    %\ra=\rightarrow
\newcommand{\ua}{\uparrow}   
\newcommand{\da}{\downarrow}   
\newcommand{\Lra}{\Leftrightarrow}          %\llra=\longleftrightarrow
\newcommand{\llra}{\longleftrightarrow}          %\llra=\longleftrightarrow

\newcommand{\stac}{\stackrel}                    %\stac=\stackrel
\newcommand{\noin}{\noindent}                    %\noin=\noindent

%Symbol for the end of the proof
\newcommand{\qed}{\nobreak\quad\vrule width6pt depth3pt height10pt}

\newcommand{\ngi}{n \ra \infty}

%heading
%\pagestyle{myheadings} \markright{Rooted edges of a minimal directed spanning tree on random points}
\pagestyle{myheadings} \markright{[Homework] Expectation and variance of sum of random variables}

\thispagestyle{plain}


\begin{center}
{\Large{\bf [Homework] Expectation and variance of sum of random variables}} 
\end{center}


\begin{enumerate}











 
 
 
 
 
 
\item
Suppose that $A$ and $B$ each randomly and independently choose 3 of 10 objects. Find the expected number of objects
\begin{enumerate}
\item chosen by both $A$ and $B$.
\\
{\color{blue}{\bf Sol.}}
Let $X$ be the number of objects that are selected by both $A$ and $B$. To further simplify
 the problem we use indicator variables $X_i$. Let $X_i = 1$ if object $i$ is selected by both $A$
 and $B$, and $X_i = 0$ otherwise, where $1\le i\le10$. Then,

$$E(X)=E(\sum_{i=1}^{10} X_i)=\sum_{i=1}^{10} E(X_i)$$

Now we must find $E(X_i)$. We know that $X_i$ only takes on one of two values, $X_i = 1$ or
 $X_i = 0$. So, for the case of a sum of independent random indicator variables, $E(X_i) =
 P(X_i = 1)$.
 Each person can choose 3 of the 10 items. There are 3 ways to choose the item of
 interest, since a person can draw 3 objects. Since person A and B draw independently,

$$P(X_i=1)=(\f{3}{10})^2$$

Then,

$$E(X)=\sum_{i=1}^{10} E(X_i)=\sum_{i=1}^{10} (\f{3}{10})^2 =\f{9}{10}$$




\item not chosen by either $A$  or $B$.
\\
{\color{blue}{\bf Sol.}}
The principle is similar to part (a). Let $X_i = 1$ if object $i$ is not chosen by $A$ and is not chosen by $B$. $P(X_i = 1) = (\f{7}{10})^2$, because the probability that an arbitrary person does
 not choose object $i$ is $\f{7}{10}$ and person A and person B draw independently. Then,

$$E(X)=\sum_{i=1}^{10} E(X_i)=\sum_{i=1}^{10} (\f{7}{10})^2 =4.9$$




\item chosen by exactly one of $A$ and $B$.
\\
{\color{blue}{\bf Sol.}}
In this case, either person $A$ draws object $i$ and person $B$ does not, or person $B$ draws object $i$ and person $A$ does not. Again, let $X_i = 1$ if exactly one of $A$ or $B$ draws object $i$, $X_i = 0$ otherwise. The person that eventually draws object $i$ had probability $\f{3}{10}$ of
 drawing the object and the person that does not draw object $i$ had probability $\f{7}
 {10}$ of not drawing object $i$. But there are two ways to arrange this situation. A can draw the
 object, and $B$ does not, or $B$ draws the object and $A$ does not. Thus,
$$E(X_i)=P(X_i=1)=2 \cdot \f{3}{10} \cdot \f{7}{10}$$
and
$$E(X)=10(2\cdot\f{3}{10}\cdot \f{7}{10})=4.2$$




\end{enumerate}
 
 
 
 
 
 \item
20 people consisting of 10 couples are in an island.
Each person lives after 1 year with probability 0.5, independently.
Let $X$ be the number of surviving couples after 1 year. 
\begin{enumerate}
\item
Identify the distribution of $X$. 



\item
Calculate the mean and variance of $X$. 



\end{enumerate}







\item
For a group of 100 people, compute
\begin{enumerate}
\item the expected number of days of the year that
are birthdays of exactly 3 people.




\item the expected number of distinct birthdays.




\end{enumerate}
 
 




\item
We have 10 different points on the unit circle.
For any two point
we flip a fair coin independently and,
if the coin lands on head, we join these two by a line segment.
Calculate the mean and variance of the number of triangles formed. 




\item
Someone I know claims to be able to spin a coin in such a way that he can make it land head 90\% of the time, on the average. I want to test the hypothesis that he's bluffing against the alternative that he is right. I propose to test this hypothesis by having him spin the coin again and again until it first lands tail. If it takes more than 4 tries, I'll conclude that he's right. Assume that the spins are independent.
\begin{enumerate}
\item
Under the null hypothesis that he cannot influence the outcome, 
identify the distribution of the number of spins until the coin lands tail.





\item
What is the expectation and variance of the number of spins to the first tail under the alternative hypothesis?


\end{enumerate}





\item
A certain project will be undertaken in 6 stages. There is a 95\% chance that each stage will be completed on time independent.
\begin{enumerate}
\item
Compute the probability that all 6 stages are completed on time.


\item
Compute the expectation and variance of the number of stages that will be completed on time.

\end{enumerate}




\item
A statistics class contains 368 students. Homework in the class is submitted online and graded automatically. The instructor wrote software to detect cheating on the homework. Suppose that the software has a 99.6\% chance of correctly identifying a student who cheats, and a 0.2\% chance of mis-identifying an honest student as a cheater. Assume that the software identification of students as cheaters or honest is independent from student to student, and that twelve of the students cheat on the homework.
\begin{enumerate}
\item A student is selected at random from the class. Compute the probability that the student cheated, given that the software says he or she did.
\\
{\color{blue}{\bf Sol.}}




\item Compute the expected number of cheaters who are correctly identified by the software.
\\
{\color{blue}{\bf Sol.}}




\item Compute the expected number of honest students who are incorrectly identified to be cheaters by the software.
\\
{\color{blue}{\bf Sol.}}




\item Compute the probability that the software correctly identifies at least ten of the cheaters.
\\
{\color{blue}{\bf Sol.}}




\item Compute the probability that the software correctly identifies all twelve of the students who cheated, without mis-identifying any of the 356 honest students to be cheaters.
\\
{\color{blue}{\bf Sol.}}




\end{enumerate}




\item
How many times would you expect to roll a fair die to see all 6 sides appeared at least once?
How about the variance?






 

\item
Consider $n$ independent flips of a fair coin. 
Say that a changeover occurs whenever an outcome differs from the one preceding it. For instance, if $n=12$ and the outcome is $HHTTTTHHHHHT$, then there are 3 changeovers.
$$
\ba{cccccccccccc}
H&H&T&T&T&T&H&H&H&H&H&T\\
&&\uparrow&&&&\uparrow&&&&&\uparrow
\ea
$$
$$
\ \ \ \ \ \ \ \ \ \ \mb{changeover}\ \ \ \ \ \mb{changeover}\ \ \ \ \ \ \ \ \ \ \ \ \mb{changeover}
$$ 
Let $X$ be the number of changeovers during the $n$ flips.
\begin{enumerate}
\item
Calculate $P(X=1)$. 


\item
Calculate the mean and variance of $X$. 




\end{enumerate}






\item
A total of $n$ balls, numbered 1 through $n$, are put into $n$ urns, also numbered 1 through $n$ in such a way that ball $i$ is equally likely to go into any of the urns $1,2,\ldots,i$. 
Find
\begin{enumerate}
\item the probability that none of the urns is empty.
\\
{\color{blue}{\bf Sol.}}
For all of the urns to have at least one ball in them, the nth ball must be dropped into
the nth urn, which has probability $\f{1}{n}$. The $n-1$st ball must be placed in the $n -1$st urn which has probability $\f{1}{n-1}$ and so on. So, the probability that none of the urns will be empty is
$$\f{1}{n}\cdot\f{1}{n-1}\cdots\f{1}{2}\cdot\f{1}{1}=\f{1}{n!}$$


\item the expected number of urns that are empty.
\\
{\color{blue}{\bf Sol.}}
Let $X$ be the number of empty urns. Define an indicator variable $X_i = 1$ if urn $i$ is empty, $X_i = 0$ otherwise. We must find $E(X_i) = P(X_i = 1)$.\\
The $i$th urn remains empty as the first $i-1$ balls are deposited into the other urns. On the $i$th drop, urn $i$ must remain empty. Since a ball can land in any of the $i$ urns with
equal probability, the probability that the ith ball will not land in urn $i$ is $1 -\f{ 1}{i}$. To
remain empty, the urn must not receive a ball on the i+1st drop etc. so the probability
that the $i + 1$st ball will not land in urn $i$, is $1-\f{1}{i+1}$. So,
\bea
E(X_i)
&=&(1-\f{1}{i})(1-\f{1}{i+1})(1-\f{1}{i+2})\cdots (1-\f{1}{n}) \nn\\
&=&(\f{i-1}{i})(\f{i}{i+1})\cdots(\f{n-1}{n})\nn\\
&=&\f{i-1}{n} \nn
\eea
\bea
E(X)
&=&\sum_{i=1}^n E(X_i)\nn\\
&=&\sum_{i=1}^n \f{i-1}{n}\nn\\
&=&\f{n-1}{2} \nn
\eea


\item the variance of the number of urns that are empty.
\\
{\color{blue}{\bf Sol.}}




\end{enumerate}







\end{enumerate}
\vskip 0.3in




\begin{center}
{\Large{\bf [Extra] Expectation and variance of sum of random variables}} 
\end{center}


\begin{enumerate}



\item
A box contains tickets labeled with the numbers $\{-3, -1, 0, 1, 3\}$. 
In 100 random draws with replacement from the box, 
calculate the expectation and the variance of the sum of the positive numbers on the tickets drawn.
[From SticiGui]





\item
Form 10 teams of 2 from 10 men and 10 women, randomly.
\begin{enumerate}
\item
Calculate the mean and variance of the number of teams of different sex. 
\item
If 20 people are actually 10 couples,
calculate the mean and variance of the number of teams of couple. 
\end{enumerate}




\item
An urn has $m$ black balls. At each stage, a black ball is removed and a new ball that is black with
probability $p$ and white with probability $1-p$ is put in its place. Find 
the expectation and variance of the number of stages needed until there are no more black balls in the urn.



\end{enumerate}



\end{document}
