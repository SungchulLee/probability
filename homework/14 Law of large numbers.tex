\documentclass[12pt]{article}%report, article
%\documentstyle[12pt,leqno]{article}

\textwidth=15.5cm \textheight=21.6cm \topmargin=-0.5cm
\oddsidemargin=0.05cm

\newcommand{\bbA}{{\bf A}}
\newcommand{\bba}{{\bf a}}
\newcommand{\bbB}{{\bf B}}
\newcommand{\bbb}{{\bf b}}
\newcommand{\bbC}{{\bf C}}
\newcommand{\bbc}{{\bf c}}
\newcommand{\bbD}{{\bf D}}
\newcommand{\bbd}{{\bf d}}
\newcommand{\bbE}{{\bf E}}
\newcommand{\bbe}{{\bf e}}
\newcommand{\bbI}{{\bf I}}
\newcommand{\bbi}{{\bf i}}
\newcommand{\bbJ}{{\bf J}}
\newcommand{\bbj}{{\bf j}}
\newcommand{\bbK}{{\bf K}}
\newcommand{\bbk}{{\bf k}}
\newcommand{\bbP}{{\bf P}}
\newcommand{\bbp}{{\bf p}}
\newcommand{\bbQ}{{\bf Q}}
\newcommand{\bbq}{{\bf q}}
\newcommand{\bbT}{{\bf T}}
\newcommand{\bbt}{{\bf t}}
\newcommand{\bbU}{{\bf U}}
\newcommand{\bbu}{{\bf u}}
\newcommand{\bbV}{{\bf V}}
\newcommand{\bbv}{{\bf v}}
\newcommand{\bbW}{{\bf W}}
\newcommand{\bbw}{{\bf w}}
\newcommand{\bbX}{{\bf X}}
\newcommand{\bbx}{{\bf x}}
\newcommand{\X}{{\cal X}}
\newcommand{\bbY}{{\bf Y}}
\newcommand{\bby}{{\bf y}}
\newcommand{\bbZ}{{\bf Z}}
\newcommand{\bbz}{{\bf z}}
\newcommand{\0}{{\bf 0}}
\newcommand{\R}{{\bf R}}
\newcommand{\txi}{\bar{\xi}}
\def\Comment#1{ \marginpar{$\bullet$\quad{\tiny #1}}}


\usepackage{graphics,graphicx,amsmath,float,color,subfigure,enumerate,booktabs}
%\usepackage[tiling]{pst-fill}
\usepackage[dvips]{xy}
\usepackage{tikz}
\usetikzlibrary{matrix}
\input{rgb}
\xyoption{all}


\newcommand{\ra}[1]{\renewcommand{\arraystretch}{#1}}


\begin{document}
%controls the margin
\baselineskip=6.0mm








%Ignore some parts of statements
\newcommand{\ignore}[1]{}{}





%Equation numbers contain section number
\renewcommand{\theequation}{\arabic{section}.\arabic{equation}}





%Activate this if I want to display eq number only
\newcommand{\lbl}{\label}

%Activate this if I want to display eq number and text number, too
%\newcommand{\lbl}[1]{\hspace{1cm} \underline{({#1})} \label{#1}}





%Call this eq numbers in text
\newcommand{\eq}[1]{$(\ref{#1})$}


\newcommand{\f}{\frac}    


%Short for Greek letters
\newcommand{\al}{\alpha}                         %\al=\al
\newcommand{\bt}{\beta}                          %\bt=w
\newcommand{\ga}{\gamma}                         %\ga=\gamma
\newcommand{\Ga}{\Gamma}                         %\Ga=\Gamma
\newcommand{\de}{\delta}                         %\de=\delta
\newcommand{\De}{\Delta}                         %\De=\Delta
\newcommand{\ep}{\epsilon}                       %\ep=\epilon
\newcommand{\ve}{\varepsilon}                    %\ve=\varepsilon
\newcommand{\la}{\lambda}                        %\la=\lambda
\newcommand{\La}{\Lambda}                        %\La=\Lambda
\newcommand{\ro}{\rho}                           %\ro=\rho
\newcommand{\ta}{\tau}                           %\ta=tau
%\newcommand{\th}{\theta}                         %\th=\theta
\newcommand{\si}{\sigma}                         %\si=\sigma
\newcommand{\Si}{\Sigma}                         %\si=\sigma
\newcommand{\om}{\omega}                           %\ro=\rho
\newcommand{\Om}{\Omega}                           %\ta=tau





%Short for equation array and for equations
%%%%%%%%%%%%%%%%%%%%%%%%%%%%%%%%%%%%
\newcommand{\be}{\begin{equation}}               %\be=\begin{equation}
\newcommand{\ee}{\end{equation}}                 %\ee=\end{equation}
%%%%%%%%%%%%%%%%%%%%%%%%%%%%%%%%%%%%
\newcommand{\bea}{\begin{eqnarray}}              %\bea=\begin{eqnarray}
\newcommand{\eea}{\end{eqnarray}}                %\eea=\end{eqnarray}
%%%%%%%%%%%%%%%%%%%%%%%%%%%%%%%%%%%%
\newcommand{\bean}{\begin{eqnarray*}}            %\beq=\begin{eqnarray*}
\newcommand{\eean}{\end{eqnarray*}}              %\eeq=\end{eqnarray*}
%%%%%%%%%%%%%%%%%%%%%%%%%%%%%%%%%%%%
%\newcommand{\beq}{\begin{eqnarray*}}            %\beq=\begin{eqnarray*}
%\newcommand{\eeq}{\end{eqnarray*}}              %\eeq=\end{eqnarray*}
%%%%%%%%%%%%%%%%%%%%%%%%%%%%%%%%%%%%
\newcommand{\ba}{\begin{array}}                  %\ba=\begin{array}
\newcommand{\ea}{\end{array}}                    %\ea=\end{array}
%%%%%%%%%%%%%%%%%%%%%%%%%%%%%%%%%%%%
\newcommand{\nn}{\nonumber}                      %\nn=\nonumber
\newcommand{\mb}{\mbox}                          %\mb=\mbox





%\newcommand{\ra}{\rightarrow}                    %\ra=\rightarrow
\newcommand{\Ra}{\Rightarrow}                    %\ra=\rightarrow
\newcommand{\ua}{\uparrow}   
\newcommand{\da}{\downarrow}   
\newcommand{\Lra}{\Leftrightarrow}          %\llra=\longleftrightarrow
\newcommand{\llra}{\longleftrightarrow}          %\llra=\longleftrightarrow

\newcommand{\stac}{\stackrel}                    %\stac=\stackrel
\newcommand{\noin}{\noindent}                    %\noin=\noindent

%Symbol for the end of the proof
\newcommand{\qed}{\nobreak\quad\vrule width6pt depth3pt height10pt}

\newcommand{\ngi}{n \ra \infty}

%heading
%\pagestyle{myheadings} \markright{Rooted edges of a minimal directed spanning tree on random points}
\pagestyle{myheadings} \markright{[Homework] Law of large numbers}

\thispagestyle{plain}


\begin{center}
{\Large\bf [Homework] Law of large numbers} 
\end{center}




\begin{enumerate}
\item
100 men and 100 women are randomly grouped into 100 pairs disjointly. Give an upper bound to the probability that at least 60 of these pairs will consist of a man and a woman.








\item
The number of items produced in a factory during a week is a random variable with mean 50.
\begin{enumerate}
\item
What can be said about the probability that this week's production will exceed 75?
\item
If the variance of a week's production is known to equal 25, then what can be said about the probability that this week's production will be between 40 and 60?
\end{enumerate}

\item
Let $X_i$ be iid $Po(1)$.
\begin{enumerate}
\item
Use Markov inequality to obtain a bound on
$P(\sum_{i=1}^{20}X_i>15)$.
\item
Use CLT to approximate
$P(\sum_{i=1}^{20}X_i>15)$.
\end{enumerate}





\item  
Let $X$ be $B(n,p)$, $n=1000$, $p=0.01$. 
Get the Chernoff bound of the probability $P(X\ge 20)$.  








\item
Show that
for a standard normal $Z$, by Chernoff's method  
$$
P(Z\ge a)\le e^{-\f{a^2}{2}}\quad \mb{for any $a\ge 0$}
$$









\item
Let $X$ be $\Ga(n,1)$.
How large need $n$ be so that
$$
P\left(\left|\f{X}{n}-1\right|>0.01\right)<0.01?
$$



\item
To determine the average lifetime of their light-emitting diode (LED) light bulbs, a manufacturer takes a simple random sample of 100 bulbs from a manufacturing lot of 32,000 bulbs. The mean lifetime of the bulbs in the sample is 102.59 thousand hours, and the sample standard deviation of their lifetimes is 8.65 thousand hours.
\begin{enumerate}
\item
Construct a conservative 95\% confidence interval for the average lifetime of their light-emitting diode (LED) light bulbs. 
\item
Construct an approximate 95\% confidence interval for the average lifetime of their light-emitting diode (LED) light bulbs. 
\end{enumerate}




\item
A lake contains 4 distinct types of fish. Suppose that each fish caught is equally likely to be any one of these types. Let $Y$ denote the number of fish that need be caught to obtain at least one of
each type.
\begin{enumerate}
\item
Give an interval $(a, b)$ such that $P(a\le Y\le b)\ge 0.90$.
\item
Using the one-sided Chebyshev inequality,
how many fish need we plan on catching so as to be at least 90 percent certain of obtaining at least one of each type.
\end{enumerate}


\item 
The lifetimes of interactive computer chips produced by a certain semiconductor manufacturer are normally distributed with parameters $\mu = 1.4 \times 10^6$ hours and $\si = 3 \times 10^5$ hours.
What is the approximate probability that a batch of 100 chips will contain at least 20 
whose lifetimes are less than
$1.8 \times 10^6$? 







\item
From past experience, a professor knows that the test score of a student taking her final examination is a random variable with mean 75 and variance 25.
\begin{enumerate}
\item
Give an upper bound for the probability that
a student's test score will exceed 85. 
\item
What can be said about the probability that a
student will score between 65 and 85?
\item
How many students would have to take the examination to ensure, with probability at least 0.9, that the class average would be within 5 of 75? Do not use the central limit
theorem.
\item
Solve (c) again using CLT.
\end{enumerate}
















\end{enumerate}


\begin{center}
{\Large\bf [Extra]} 
\end{center}






\begin{enumerate}


\item
Suppose that the number of typographical errors on a single page of a particular book is $Po(\la)$, $\la=1$. Calculate 
\begin{enumerate}
\item
the probability that there is no error on one particular page. 
\item
the probability that there is at most two errors on next 10 pages. 
\item
the probability that there is at most 200 errors in this book of 1000 pages.
Use the Markov inequality and calculate the upper bound of this probability. 
\item
the probability that there is at most 200 errors in this book of 1000 pages.
Use the Chebyshev inequality and calculate the upper bound of this probability. 
\item
the probability that there is at most 200 errors in this book of 1000 pages.
Use the one-sided Chebyshev inequality and calculate the upper bound of this probability.
\item
the probability that there is at most 200 errors in this book of 1000 pages.
Use the Chernoff bound and calculate the upper bound of this probability.
\item
the probability that there is at most 200 errors in this book of 1000 pages.
Use the CLT and approximately calculate this probability. 
\item
the expectation and variance of the page number that there is an error for the first time when we examine the book from page 1.
In this computation we assume that this book has an infinite number of pages for simplicity. 
\item
the expectation and variance of the page number that we find an error for the first time when we examine the book from page 1,
under the assumption that we find the error with probability 40\% when we encounter.
In this computation we assume that this book has an infinite number of pages for simplicity. 
\end{enumerate}




\item
Let $X_i$ be iid with $P(X_i=1)=P(X_i=-1)=\f12$.
Calculate the Chernoff bound of $P(S_{10}\ge 6)$, where $S_{10}=\sum_{i=1}^{10}X_i$.



\item
Show that 
for $X\sim Po(\la)$, by Chernoff's method 
$$
P(X\ge n)\le e^{-\la}\left(\f{e\la}{n}\right)^n\quad \mb{for any non-negative integer $n$}
$$





\item
The entering class at a certainUniversity is about 1000 students. The dean's office seeks to determine the percentage of entering students who have credit cards. The dean's office will take a simple random sample of 35 entering students, interview them, and compute the sample percentage. 
The sample is taken, and the sample percentage is observed to be 50\%.
\begin{enumerate}
\item
Construct a conservative 95\% confidence interval for the percentage of entering students who have credit cards. 
\item
Construct an approximate 95\% confidence interval for the percentage of entering students who have credit cards. 
\end{enumerate}




\end{enumerate}







\end{document}
